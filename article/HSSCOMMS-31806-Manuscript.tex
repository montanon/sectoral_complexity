\documentclass[10pt]{article}
\usepackage[english]{babel}
\usepackage[a4paper,top=2cm,bottom=2cm,left=1.5cm,right=1.5cm,marginparwidth=1.75cm]{geometry}

\usepackage{natbib}
\setcitestyle{authoryear,open={(},close={)}}

\usepackage{amsmath}
\usepackage{graphicx}
\usepackage[colorlinks=true, allcolors=blue]{hyperref}
\usepackage{lipsum} 
\usepackage{adjustbox}
\usepackage{booktabs}
\usepackage{multirow}
\usepackage{multicol}
\usepackage{caption}
\usepackage{float}
\usepackage{graphicx} 
\usepackage{xltabular}
\usepackage[title]{appendix}
\usepackage{authblk}
\usepackage{etoolbox}

\newcommand{\HI}{\textit{High income}}
\newcommand{\UM}{\textit{Upper middle income}}
\newcommand{\LM}{\textit{Lower middle income}}  
\newcommand{\LI}{\textit{Low income}}

\newcommand{\AG}{\textit{Agriculture}}
\newcommand{\EL}{\textit{Electronics~\& Instruments}}
\newcommand{\FI}{\textit{Fishing}}
\newcommand{\FO}{\textit{Food~\& Beverages}}
\newcommand{\IR}{\textit{Iron~\& Steel}}
\newcommand{\MA}{\textit{Machinery}}
\newcommand{\ME}{\textit{Metal Products}}
\newcommand{\MI}{\textit{Mining~\& Quarrying}}
\newcommand{\OT}{\textit{Other Manufacturing}}
\newcommand{\PE}{\textit{Petroleum, Chemicals~\& Non-Metals}}
\newcommand{\TE}{\textit{Textiles~\& Wearing Apparel}}
\newcommand{\WO}{\textit{Wood~\& Paper}}

\title{Sectoral economic complexity and environmental degradation: a sectoral perspective on the EKC hypothesis}
\author[1]{Sebastian Montagna}
\author[1]{Liqiao Huang}
\author[1]{Yin Long}
\author[1]{Yoshikuni Yoshida}

\affil[1]{Graduate School of Engineering, University of Tokyo, Tokyo, Japan\\
\texttt{montagna-puga-sebastian@g.ecc.u-tokyo.ac.jp},\quad
\texttt{liqiaohuang@g.ecc.u-tokyo.ac.jp},\quad
\texttt{long@tmi.t.u-tokyo.ac.jp},\quad
\texttt{y-yoshida@e.t.u-tokyo.ac.jp}}

\makeatletter
\newsavebox{\mytitlebox}
\AtBeginDocument{
  \sbox{\mytitlebox}{\parbox{\textwidth}{
    \centering
    {\LARGE \@title \par}
    \vspace{1em}
    \@author
  }}
}
\makeatother

\begin{document}
\maketitle

\begin{abstract}

As rising CO\textsubscript{2} emissions drive global environmental concerns, addressing climate change and ensuring sustainable development requires a nuanced understanding of the relationship between economic development and environmental degradation. This study introduces the Sectoral Complexity Index (SCI) to measure and analyze the sophistication of individual economic sectors and examines their influence on CO\textsubscript{2} emissions across 127 countries from 1995 to 2020. By refining the Environmental Kuznets Curve (EKC) hypothesis through a sectoral lens, the research uses a cross-sectional quantile regression to capture sector-specific environmental dynamics at varying stages of economic development. The results reveal heterogeneous patterns across sectors and income groups. Key industries, such as \IR, \MA, \ME, and \MI\ show reduced CO\textsubscript{2} emissions with increased sophistication. Notably, the \IR\ and \MA\ sectors exhibit a strong transition to lower emissions at the upper-middle-income level, while such transitions in the \ME\ and \MI\ sectors occurs at the high-income level. These results underscore the sectoral and income-specific dynamics of the environment-economy relationship, highlighting the importance of targeted policies that promote technological innovation and green energy transition in energy-intensive sectors. By offering deeper insights into the interplay between economic development and environmental sustainability, this research contributes to the discourse on the environment-economy relationship and informs policy strategies aimed at fostering sustainable economic development.

\end{abstract}

\section{Introduction}
Over recent decades, environmental concerns have risen to the forefront of international discourse, with climate change emerging as a critical issue. International initiatives, such as the Paris Agreement \citep{paris}, the Intergovernmental Panel on Climate Change \citep{IPCC_2022_WGIII}, and the Sustainable Development Goals \citep{sdgs}, have set ambitious targets for reducing greenhouse gas emissions, particularly carbon dioxide (CO\textsubscript{2}) \citep{Friedlingstein2010Update}. However, the lack of consensus over responsibility allocation between developed and developing nations has impeded collective progress \citep{ari2017differentiation}.

The underlying challenge lies in the intricate interplay between economic development and environmental degradation. Extensive research has examined this nexus \citep{al2015investigating}, indicating that the environmental impacts of economic activity evolve as nations advance through different stages of economic development \citep{dougan2022relevance}. In earlier stages of development, economic expansion is often underpinned by lax environmental regulations, heavy reliance on low-cost energy sources, and the deployment of resource-intensive technologies---factors that contribute to accessible and rapid economic growth, yet at substantial environmental cost \citep{alvaradoEcologicalFootprintEconomic2021}.

Conversely, more advanced developmental stages often exhibit a relative decoupling of economic growth from environmental degradation, fostered by cleaner technology adoption, stricter regulations, and a gradual shift toward service-oriented and knowledge-based economic activities \citep{Wegrzyn2013The}. This divergence in developmental trajectories underscores the persistent difficulties in achieving a global consensus on environmental responsibilities: developing nations may be reluctant to adopt stringent environmental policies that could constrain their economic development. In contrast, developed nations have already navigated the environment-intensive phase \citep{zhangComparingDevelopedEmerging2024}.

Furthermore, pursuing economic development remains a critical driving force in policy agendas across developed and developing nations, often overshadowing long-run sustainability considerations \citep{eisenmengerSustainableDevelopmentGoals2020}. On the one hand, robust economic performance fosters employment, draws foreign investments, and bolsters international competitiveness \citep{bajajUnleashingEconomicPotential2024}. On the other hand, prioritizing growth objectives may detract from adopting or enforcing rigorous environmental standards, as industries and policymakers perceive these regulations as costly and potentially detrimental to short-term economic metrics \citep{xuTradeoffEnvironmentEconomy2023}.

The resulting tensions typify the intrinsic dilemma embedded within global sustainability agendas, where nations must continually balance the trade-offs between national economic competitiveness and environmental stewardship while adhering to international commitments.

The intricacies of international competition further intensify these issues. Stringent environmental regulations in one jurisdiction might prompt industries to relocate to regions with more lenient standards---a phenomenon known as the pollution haven hypothesis, which posits that multinational firms relocate production to countries with lower environmental compliance costs \citep{Ma_Shi_2023}. Conversely, the pollution halo hypothesis suggests that a multinational firm's direct investment may transfer cleaner technologies and more sophisticated practices and knowledge, generating a halo effect that improves environmental capabilities in the host nation over time \citep{abbassDoesTechnologyInnovation2022}.

These hypotheses highlight the relevancy of fostering international frameworks and sustainable relationships between foreign investors and local institutions and reinforcing focused capability-building mechanisms \citep{olabisiDesigningEffectivePolicy2024}. Such policies underscore the importance of transparent and accountable local governance safeguarding equitable, non-exploitative partnerships \citep{mudambiGovernmentRegulationCorruption2013}. By prompting such knowledge transfer in developing regions, multinational enterprises can catalyze local economic sophistication and the eventual decoupling of economic activity from environmental degradation \citep{fuMultinationalEnterprisesStructural2021}.

A pertinent example of this dynamic can be observed in China's industrial evolution. In its rapid ascent to become the "world's factory," China relied heavily on foreign firms' investment in manufacturing and coal---benefiting from inexpensive and abundant energy sources---to develop a robust manufacturing and infrastructure base \citep{pengMeasurementDrivingFactors2022, guoAssessingRelationshipForeign2024}. This coal-based trajectory, however, also contributed to significant environmental challenges, ranging from severe air pollution to elevated carbon emissions \citep{ZHOU2019793}.

In more recent years, recognizing the long-term costs of extensive coal dependence, China has strategically pivoted toward renewable energy investments and high-end green technologies \citep{ChinaGreenTechnologies}, leveraging its earlier industrial capability-building to develop novel domestic capabilities in solar and wind power production \citep{wangGreenTechnologyInnovation2019}, among others. This transition exemplifies how early, resource-intensive phases of development, led by robust governance and targeted policy frameworks, can serve as a launchpad for subsequent adoption of cleaner practices and overall economic-environmental decoupling \citep{balsa-barreiroGlobalizationShiftingCenters2019}. Although this shift demonstrates how an initial resource-intensive phase may serve as a platform for a green transition, the substantial environmental toll incurred along the way offers a cautionary lesson for international governance and other emerging economies seeking economic growth that goes hand in hand with environmental protection \citep{vennemoEnvironmentalPollutionChina2009}.

The Environmental Kuznets Curve (EKC) hypothesis provides a valuable theoretical lens for interpreting and examining this complex relationship between economic development and environmental degradation \citep{GrossmanEKCKrueger}. The EKC posits that the environment-economy relationship follows an inverted U-shaped relationship, where environmental degradation intensifies during initial economic development but then declines as economies mature, income levels rise, and economies adopt cleaner practices \citep{paoMultivariateGrangerCausality2011, hanifFossilFuelsForeign2019, leitaoImpactRenewableEnergy2021}.

Unlike simpler linear models that assume a monotonic relationship, the EKC offers a structured framework for conceptualizing the non-linear trajectory that countries may follow as they progress economically. For policymakers and stakeholders who must balance economic endeavors with sustainability goals, it is a helpful heuristic that points to the crucial role of proactive governance, institutional efficacy, and technological innovation in enabling transitions to greener pathways  \citep{hussainCurvatureTurningPoint2023, EKCpolicy}.

However, while the EKC hypothesis offers a valuable conceptual framework for examining the national interplay between economic development and environmental degradation, effectively achieving such decoupling requires more nuanced governance interventions that address the multilayered interactions driving overall behavior. These interventions must integrate national priorities with international commitments, robust governance structures, and transparent institutions, foster capacity-building, and reconcile short- and long-term trade-offs \citep{katramizAccelerating2030Agenda2021}.

Moreover, the diversity in environmental, economic, and developmental conditions observed across nations underscores the importance of understanding the complexity of each country's economy to achieve tailored policy interventions. In this regard, Economic Complexity, as conceptualized by \cite{hidalgoBuildingBlocksEconomic2009}, provides a framework to quantify the capabilities embedded within a nation's productive structure.

Unlike traditional metrics such as GDP, which primarily measure the scale of economic activity, the Economic Complexity Index (ECI) captures the know-how and productive capabilities by measuring not just \textit{how much} a country produces, but \textit{what} it produces \citep{hidalgoBuildingBlocksEconomic2009}. The underlying principle is that complex products require more diverse and specialized capabilities, and only sophisticated countries with more robust institutions, specialized technological capacities, and broader knowledge networks produce them \citep{hidalgoProductSpaceConditions2007}. In contrast, countries that export primarily simple or ubiquitous products have less sophisticated capabilities and thus rank lower on the complexity scale.

Economic complexity addresses the inherent heterogeneity in development pathways by providing a nuanced measure of economic progress beyond income levels that offers insights into why countries with similar GDP per capita may exhibit vastly different environmental impacts. For instance, nations with higher economic complexity often demonstrate a greater capacity to adopt cleaner technologies, enforce stringent environmental regulations, and transition toward knowledge-based and service-oriented economic activities. In contrast, less complex economies rely on resource-intensive industries that exacerbate environmental degradation \citep{doganDoesEconomicComplexity2019}. 

Thus, the EKC offers a structured narrative on the non-linear dynamics of environmental degradation along economic development pathways, and the ECI provides a detailed lens to evaluate the structural capabilities underpinning such transitions. 

Integrating these frameworks allows for a deeper understanding of why some nations achieve earlier or more pronounced decoupling of economic growth from environmental degradation, emphasizing the role of productive sophistication and institutional capacity \citep{degirmenciDecouplingSustainableDevelopment2024}. By leveraging the strengths of both models, this combined approach can illuminate the nuanced mechanisms through which economic complexity mediates the trajectory of environmental sustainability, particularly in the context of achieving targeted and effective governance interventions. 

Indeed, several studies have integrated these approaches by testing the EKC hypothesis with the ECI as the developmental indicator and have found meaningful results \citep{neaguLinkEconomicComplexity2019, yilanciInvestigatingEKCHypothesis2020}. However, an important research gap remains regarding the \textit{sub-national} or \textit{sectoral-level} mechanisms driving this relationship. Existing literature has predominantly focused on national aggregates, overlooking that structural differences across sectors may yield unique environmental trajectories within the same country, and there is potential to dig further by leveraging another branch of research on sub-national Economic Complexity indexes.

The present study introduces the Sectoral Complexity Index (SCI) as a more granular measure of productive sophistication to test the EKC hypothesis at the sectoral level. Specifically, we investigate the following questions: \textit{(i) How does sector-level economic complexity affect per capita CO\textsubscript{2} emissions across countries at different stages of economic development? (ii) Which economic sectors drive the national economy-environment relationship? (iii) What are the implications for sustainable development policies?} 

To capture the heterogeneous effects across different countries, we employ a quantile regression analysis of data from 127 countries spanning 1995 to 2020. Through this lens, we uncover complex patterns that enrich the traditional EKC hypothesis, revealing variability in the relationship between each industrial sector's complexity and per capita CO\textsubscript{2} emissions. The study further delineates distinct developmental pathways by stratifying the analysis based on income levels, thereby contributing to a more nuanced understanding of sustainable development.

By recognizing the complexity of the economy-environment nexus, policymakers and scholars can better devise strategies to advance economic development while mitigating environmental impacts, ultimately steering development pathways toward long-term sustainability. The remainder of this paper is structured as follows: \textit{Section~2} elaborates the theoretical framework and literature review, \textit{Section~3} details the data and methodology, \textit{Section~4} presents the results and discussions, and \textit{Section~6} concludes by reflecting on the policy implications, future research directions and the broader significance of these findings for sustainable development.

\section{Literature Review}

\cite{GrossmanEKCKrueger} examined the relationship between income per capita and urban air pollutants across 42 countries, revealing that air pollution increases with income at lower income levels but begins to decline as income rises in high-income economies. This pattern suggests a threshold developmental level where economic growth contributes to alleviating air pollution. Building on these findings, \cite{panayotouEnvironmentalDegradationDifferent1995} formalized the concept of an inverted U-shaped relationship between environmental degradation and economic development, coining it the "Environmental Kuznets Curve." Empirical validation of this hypothesis demonstrated that economies undergo structural transitions at different developmental stages, highlighting the critical role of effective policy interventions in mitigating environmental degradation.

Subsequent research has further validated the EKC hypothesis, establishing it as a cornerstone in environmental economics discussions \citep{lealEvolutionEnvironmentalKuznets2022}. For instance, \cite{adzawlaGreenhouseGassesEmission2019} confirmed the EKC relationship between CO\textsubscript{2} emissions and GDP in sub-Saharan Africa using aggregated panel data from 1970 to 2012. Similarly, \cite{xiaInvestigatingRoleGlobalization2022} examined 67 developing and developed countries, incorporating energy consumption and globalisation, as measured by the KOF Globalisation Index \citep{KOFGlobalisationIndex}. Their findings supported the EKC hypothesis, revealing a positive relationship between CO\textsubscript{2} emissions and globalization, emphasizing the interconnectedness of economic and environmental dimensions at the national and international levels.

\cite{paoMultivariateGrangerCausality2011} explored the relationships between CO\textsubscript{2} emissions, energy consumption, foreign direct investment (FDI), and GDP in BRIC countries through panel cointegration and multivariate Granger causality analysis. Their findings validated the EKC hypothesis and supported the pollution haven hypothesis, identifying halo and scale effects. These results underscore the importance of effective capability transfer between multinational corporations and host nations. The study further suggested that BRIC countries could reduce CO\textsubscript{2} emissions without compromising competitiveness by integrating FDI management with energy efficiency and energy supply investments.

Despite its widespread acceptance, studies have challenged the universality of the EKC hypothesis by revealing more intricate relationships between economic development and environmental degradation \citep{purcelNewInsightsEnvironmental2020}. For example, \cite{lazarPollutionEconomicGrowth2019} observed a monotonically positive relationship between pollution and economic growth in 11 Central and Eastern European countries, contradicting the EKC hypothesis. Similarly, \cite{nasirRoleFinancialDevelopment2019} found a statistically significant positive relationship between economic development and CO\textsubscript{2} emissions while failing to validate the EKC hypothesis. \cite{ozcanNexusCarbonEmissions2013} found that among 12 Middle Eastern countries, only three confirmed the EKC hypothesis, while others displayed U-shaped or insignificant relationships. Furthermore, \cite{ozokcuEconomicGrowthEnergy2017} extended this analysis to two groups comprising 26 high-income countries and 52 emerging countries and observed N-shaped and inverted N-shaped relationships. These results underscore the variability and complexity of the environment-economy nexus and the heterogeneity of developmental pathways.

Subsequent research has adopted broader measures of environmental impact, such as the Ecological Footprint \citep{ecofootprint, destekInvestigationEnvironmentalKuznets2019}, and incorporated more nuanced economic indicators, including the ECI, to address the EKC limitations. The ECI, which quantifies the sophistication of an economy's productive structure based on export composition, provides a deeper understanding of economic dynamics \citep{hidalgoBuildingBlocksEconomic2009}. Studies have shown that ECI outperforms traditional indicators such as GDP per capita in predicting long-term economic growth \citep{hidalgoEconomicComplexityTheory2021}, emphasizing the significance of country-specific characteristics \citep{zhuEconomicComplexityHuman2017, guoTradeOpennessGlobalization2023}. By capturing latent capabilities such as knowledge intensity and technological sophistication, the ECI offers valuable insights into a country's economic structure and its implications for economic development \citep{hidalgoEconomicComplexityTheory2021}.

Furthermore, the ECI and the associated product space also capture structural changes at the per-product level \citep{hidalgoProductSpaceConditions2007}, allowing policymakers to identify strategic sectors where technological advancements and targeted policies can drive resource-efficient economic improvements \citep{hidalgoEconomicComplexityTheory2021}. Additionally, the ECI strongly correlates with environmental outcomes \citep{agozieEnvironmentalKuznetsCurve2022, alvaradoEcologicalFootprintEconomic2021, canEffectExportComposition2022}, positioning it as a valuable tool for testing and refining the EKC hypothesis. For instance, \cite{doganDoesEconomicComplexity2019} integrated the ECI into their analysis of the EKC hypothesis and found that economic complexity significantly influences the economy-environment relationship. Their results revealed that increases in the ECI are associated with higher CO\textsubscript{2} emissions in lower- and upper-middle-income countries but lead to a reduction in CO\textsubscript{2} emissions in high-income countries. Similarly, \cite{youDynamicImpactEconomic2022} and \cite{khezriEnvironmentalImplicationsEconomic2022} validated the EKC hypothesis using the ECI as a proxy for economic development in their analyses of 95 countries from 1996 to 2015 and 29 Asia-Pacific countries from 2000 to 2018, respectively.

As economies transition toward more complex manufacturing and service sectors, they increasingly adopt cleaner technologies, diversify energy sources, enhance efficiency, and implement stricter environmental regulations. These factors collectively mitigate environmental pressures \citep{Ahmad2023Natural, ullahExploringRoleGreen2024}. Consequently, integrating ECI into sustainability frameworks provides a valuable approach to aligning economic development objectives with environmental imperatives.

Another approach to refine the EKC hypothesis involves sectoral analysis, which examines the relationship between economic development and environmental degradation within specific economic sectors. This method recognizes that the environmental impacts of economic activity vary significantly across industries due to differences in production processes, energy consumption, and technological adoption. Each economic sector contributes differently to economic development and environmental degradation and interacts with the other sectors. 

For example, manufacturing sectors have high energy consumption and emissions levels due to resource-intensive production processes. In contrast, service sectors generally have a lower direct environmental footprint but may contribute indirectly to environmental impacts through increased demand for energy, transportation, and consumer goods \citep{Madlberger2024}. Agriculture presents distinct environmental challenges, including deforestation, methane emissions, and excessive water usage, which differ from those of other sectors \citep{omotosoInterplayAgricultureGreenhouse2024}. Similarly, the transport sector significantly contributes to greenhouse gas emissions, primarily due to its reliance on fossil fuels and infrastructure development. It also serves as a critical enabler for other sectors and plays a pivotal role in economic development \citep{shafiqueInvestigatingNexusTransport2021}. These sector-specific characteristics substantially shape the trajectory of environmental degradation as economies progress.

Focusing on sector-specific dynamics allows researchers to uncover patterns of environmental degradation that may be overlooked in aggregate analyses, thereby offering more granular insights into the applicability of the EKC hypothesis. This sectoral perspective deepens the understanding of the environment-economy relationship. It equips policymakers with the tools to design further targeted interventions that address the distinct environmental challenges posed by each sector \citep{sectoralEKCandpolicy}.

\cite{azizRevisitingRoleForestry2020} investigated the EKC hypothesis in Pakistan, using the Ecological Footprint as the environmental indicator. Employing a quantile autoregressive distributed lag (QARDL) approach, their study identified a quantile-dependent relationship between economic growth, forest area, agriculture, renewable energy, and environmental degradation. The findings validated the EKC hypothesis and demonstrated that renewable energy consumption, forest preservation, and eco-friendly agricultural technologies significantly reduce the Ecological Footprint. Similarly, \cite{omriEntrepreneurshipSectoralOutputs2018} empirically analyzed the relationship between CO\textsubscript{2} emissions, entrepreneurship, and the value-added of the agriculture, industrial, and service sectors across 69 countries categorized by income levels. The results supported the inverted U-shaped relationship proposed by the EKC in low-, lower-middle-, and upper-middle-income countries. They revealed a similar inverted U-shaped relationship between entrepreneurship and environmental degradation.

Furthermore, \cite{moutinhoDeterminantsEnvironmentalKuznets2020} examined the EKC hypothesis in the context of 12 OPEC countries, focusing on the role of economic sector diversification in the environment-economy relationship. The study identified a U-shaped relationship between economic development and environmental degradation in all countries using annual data from 1992 to 2015 and applying panel cross-section analysis with corrected standard errors and convergence estimations across seven sectors. It emphasized the importance of sector-specific investments in renewable energy and efficiency improvements, highlighting pathways for mitigating environmental harm in oil-dependent economies.

As with the EKC, researchers have also extended the ECI framework at the sectoral level. For instance, \cite{yeCanIncreasedEconomic2023} examined the relationship between Economic Complexity and emissions from the logistics industry in Belt and Road countries. Demonstrating that technological innovation, as captured by the ECI, provides valuable insights and supports the EKC hypothesis within this sector. Similarly, \cite{taghvaeeEconomicComplexityCO22022} validated the EKC hypothesis by enhancing empirical models with data on emissions from the agriculture, industry, and service sectors in OECD countries, revealing distinct behaviors among these sectors. Such sector-specific analyses highlight that the relationship between economic development and environmental degradation varies across development stages and economic sectors.

Moreover, by building on the granularity of the Economic Complexity framework, researchers have extended its application to measure economic complexity at the sub-national level. \cite{wangCanIncreasingEconomic2023} examined the relationship between green development efficiency (GDE) and economic complexity across Chinese provinces, revealing significant regional disparities. The study highlighted the mediating roles of innovation and human capital in improving GDE through enhanced economic complexity. Similarly, \cite{lapatinasEconomicComplexityCities2022} developed an index to measure the economic complexity of cities globally, providing policymakers with a tool to track urban economic outcomes. The study found that economically complex cities are highly diversified and host unique global firm activities using data from the world's largest firms and their subsidiaries. An analysis of EU cities indicated that higher economic complexity correlates with greater prosperity, innovation, and infrastructure. Furthermore, the index predicted urban resilience, showing that more complex cities recover employment more rapidly following economic shocks, underscoring the importance of diversification and sophistication.

Most notably, \cite{mealyEconomicComplexityGreen2022} introduced the Green Complexity Index (GCI) to evaluate countries' sophistication in the green economy. By constructing a comprehensive dataset of traded green products and applying economic complexity methodologies, the study assessed countries based on their ability to export complex green products competitively and their environmental performance. The findings demonstrated that countries with higher GCI scores exhibit more substantial environmental innovation, lower CO\textsubscript{2} emissions, and more stringent environmental policies independent of their per capita GDP. Additionally, the study investigated the potential for countries to transition into new green product markets, revealing significant path dependence in accumulating green capabilities. These findings enhance the understanding of the dynamics underpinning the green economy development and offer valuable insights for sectoral analyses and the formulation of effective green industrial policies.

Despite extensive research and significant progress in understanding the relationship between economic development, environmental degradation, and economic complexity, several research gaps remain. While the EKC hypothesis has been explored extensively at the aggregate and sectoral levels, its integration with the ECI and related frameworks remains limited, particularly in capturing the intricate interplay between economic complexity, sector-specific dynamics, and environmental outcomes. Furthermore, although the extension of the ECI framework to sub-national and sectoral levels has revealed promising insights, its potential to capture the relationship between sectoral economic sophistication and environmental degradation remains underexplored. Novel indices such as the GCI offer valuable perspectives into economies' structures by examining specific sectors but fail to provide a comprehensive view of the broader economy and the intersectoral dynamics. These gaps highlight the need for further research combining the EKC and ECI frameworks with thorough sectoral and environmental dimensions to provide more granular insights into the heterogeneous pathways of sustainable development across different economies and actionable insights for policymakers.

This study aims to enhance the comprehension of the relationship between environmental degradation and economic development by introducing a novel Sectoral Complexity Index (SCI) encompassing all productive sectors underlying the national-level ECI. Building on \cite{mealyEconomicComplexityGreen2022} work, we classify all 5,011 export products into 12 economic sectors and calculate a sectoral-level economic complexity index that, as with the GCI for the green economy, captures the degree of sophistication in the respective sectors for 127 countries from 1995 to 2020. This study uses the SCI to test the EKC hypothesis for all sectors through quantile regression analysis on CO\textsubscript{2} emissions per capita. By further stratifying by income levels, the analysis delineates three factors: economic development on the income level axis, environmental degradation on the CO\textsubscript{2} axis, and the SCI of sectors driving the relation between them and offering a more granular understanding of sustainable development. The findings challenge and refine the traditional EKC hypothesis, showing that sustainable development milestones are highly path-dependent with complex interactions derived from economic sectors' sophistication at different stages of economic development. These insights underscore the need for policy frameworks that account for the unique roles of economic sectors, emphasizing that a sectoral perspective is critical for addressing the diverse challenges and opportunities economies face in achieving sustainable development.

\section{Methods}
\subsection{Data}
\label{sec:data-cvars}
The dataset used in this study compromises export data from The Observatory of Economic Complexity (OEC) \citep{TheOEC}. The data is classified according to the Reviewed Harmonized System (HS6) classification at the 6-digit level, encompassing 5,011 products reported by 142 countries from 1995 to 2020. Data cleaning follows the criteria outlined in \cite{hidalgoBuildingBlocksEconomic2009} for a consistent and robust analysis. Specifically, we restrict the analysis to countries with a minimum population of 1 million and export values exceeding 1 billion USD as of 2020. These criteria yield a final dataset of 127 countries. A comprehensive list of the included countries and their corresponding income classifications is available in Supplementary Section SD.

Additional data, including \textit{CO\textsubscript{2} emissions (metric tons per capita)}, \textit{Energy Consumption (measured in million tonnes of oil equivalent, Mtoe)}, and \textit{Environmental Patents} were obtained from the OEC. Data on \textit{Renewable energy consumption (as \% of total final energy consumption)}, \textit{Total natural resources rents (as \% of GDP)}, \textit{Urban population (as \% of total population)}, \textit{Income level}, and \textit{Total population} were sourced from the World Bank. We calculate per capita values for \textit{Energy Consumption} and \textit{Environmental Patents} by dividing the respective national totals by the population of each country. The analysis also includes the \textit{Economic Globalisation Index} \citep{KOFGlobalisationIndex}.

We then group the 5,011 products in the HS6 classification into twelve economic sectors through an iterative process. This process began by leveraging existing product-sector mappings from Input-Output Tables \citep{lenzenMappingStructureWorld2012}. We then manually refine the initial classification to enhance sector expressiveness, reduce multicollinearity, and improve Community Quality, as defined by \cite{hausmann2014atlas}. The final classification, summarized in Table \ref{tab:sector-products}, presents the twelve sectors and their respective product counts. A comprehensive and detailed classification is available in Supplementary Section SG.


\begin{table}[ht]
      \centering
      \caption{Number of Products by Sector.}
      \begin{tabular}{lr}
            \toprule
            & N°\ Products \\
 Sector &  \\
            \midrule
 Agriculture & 445 \\
 Electronics\ \&\ Instruments & 453 \\
 Fishing & 85 \\
 Food\ \&\ Beverages & 252 \\
 Iron\ \&\ Steel & 308 \\
 Machinery & 577 \\
 Metal\ Products & 296 \\
 Mining\ \&\ Quarrying & 163 \\
 Other\ Manufacturing & 273 \\
 Petroleum,\ Chemicals\ \&\ Non-Metals & 1071 \\
 Textiles\ \&\ Wearing\ Apparel & 851 \\
 Wood\ \&\ Paper & 237 \\
            \bottomrule
      \end{tabular}
      \label{tab:sector-products}
\end{table}

\subsection{Economic Complexity and Product Complexity}
We calculate the ECI for countries and the Product Complexity Index (PCI) for products using the methodology established in \cite{hidalgoBuildingBlocksEconomic2009} and later refined in matrix form by \cite{mealyInterpretingEconomicComplexity2019}.

The annual export data is expressed as matrix $x_{cp}$, where rows correspond to countries $c$, columns to products $p$, and each entry reflects the export value of product $p$ by country $c$. We then compute the corresponding RCA matrix values for each $c,p$ pair using the formulation of the Balassa Index \citep{BalassaIndex} in equation \eqref{eqn:rca}.

\begin{equation}\label{eqn:rca}
 RCA_{cp} = \frac{\frac{x_{cp}}{\sum_{p}x_{cp}}}{\frac{\sum_{c}x_{cp}}{\sum_{c}\sum_{p}x_{cp}}}
\end{equation}

In this formulation, $x_{cp}$ represents the matrix of exports, $\sum_{p}x_{cp}$ denotes the total exports of each country, $\sum_{c}x_{cp}$ represents the total global exports of each product, and $\sum_{c}\sum_{p}x_{cp}$ corresponds to the total global exports across all products and countries. Thus, the $RCA_{cp}$ is defined as a ratio that compares the share of product $p$ in the exports of country $c$ to the share of product $p$ in global exports, providing a measure of the relative importance of a product within a country's export structure.

Given the heavy-tailed distribution of $RCA_{cp}$ values with extreme positive outliers, the standard procedure outlined by \cite{hidalgoEconomicComplexityTheory2021} is adopted. A threshold of $RCA_{cp} = 1$ is applied to binarize the $RCA_{cp}$ matrix, resulting in the specialization matrix $M_{cp}$. This matrix contains binary values, where $M_{cp} = 1$ indicates that country $c$ specializes in product $p$, and $M_{cp} = 0$ otherwise, as defined in equation \eqref{eqn:mcp}.

\begin{equation}\label{eqn:mcp}
 M_{cp} = \left\{\begin{matrix}
            1, & RCA_{cp} \geq 1\\ 
            0, & RCA_{cp} < 1
      \end{matrix}\right.
\end{equation}

The row-wise summation of the specialization matrix $M_{cp}$ yields the diversity vector, denoted as $d_{c}$, as defined in equation \eqref{eqn:diversity}. This vector indicates the number of products in which each country $c$ specializes. Similarly, summing across the columns of $M_{cp}$ produces the ubiquity vector, $u_{p}$, as shown in equation \eqref{eqn:ubiquity}, which reflects the number of countries that specialize in each product $p$.

\begin{equation}\label{eqn:diversity}
 d_{c} = \sum_{p}M_{cp}
\end{equation}
\begin{equation}\label{eqn:ubiquity}
 u_{p} = \sum_{c}M_{cp}
\end{equation}

The corresponding $ECI_{c}$ vector, with the ECI value for each country, is derived from the previously computed matrices. We follow the methodology outlined in \citep{mealyInterpretingEconomicComplexity2019} to calculate the ECI as the eigenvector associated with the second-largest eigenvalue of the matrix defined by equation \eqref{eqn:m-tilda}.

\begin{equation}\label{eqn:m-tilda}
 \widetilde{M} = D^{-1}S
\end{equation}

In this formulation, $D$ represents the diagonalized matrix of the $d_{c}$ diversity vector, while $S$ corresponds to the symmetric similarity matrix of the pair-wise comparison between countries' export specializations, as computed from equation \eqref{eqn:s-matrix}.

\begin{equation}\label{eqn:s-matrix}
 S_{cc'} = \sum_{p}\frac{M_{cp}M_{c'p}}{u_{p}}
\end{equation}

Similarly, the associated $PCI_{p}$ vector, which contains the PCI values of each product, is derived as the eigenvector corresponding to the second largest eigenvalue of $\widetilde{M}^{T}$, where $T$ denotes the transpose operation. These eigenvector computations lead to the solution of a set of decoupled self-consistent equations, as defined by the relation presented in equation \eqref{eqn:eci-pci}, initially introduced in \cite{hidalgoBuildingBlocksEconomic2009}.

\begin{equation}\label{eqn:eci-pci}
 ECI_{c} = \frac{1}{d_{c}}\sum_{p}M_{cp}PCI_{p}
\end{equation}

Finally, since the ECI and PCI values represent relative metrics, the vectors are standardized by subtracting their mean and dividing by their standard deviation, following the procedure outlined in \cite{hidalgoBuildingBlocksEconomic2009}.

\subsection{Sectoral Complexity Index}
\label{sec:sci}

Building on the computed PCI values for each product and the predefined economic sectors, the methodology introduced by \cite{mealyEconomicComplexityGreen2022} for calculating the GCI is adapted to derive the Sectoral Complexity Index (SCI), as specified in equation \eqref{eqn:sci}.

\begin{equation}\label{eqn:sci}
 SCI_{cs} = \sum_{p\epsilon s}RCA_{cp}\cdot PCI_{p}^{s}
\end{equation}

In this formulation, $RCA_{cp}$ denotes the binary RCA value for the corresponding $c,p$ pair from the matric $M_{cp}$, while $PCI_{p}^{s}$ represents the positively-defined PCI values for product $p$ within sector $s$, as specified in equation \eqref{eqn:pci-plus}.

This step is necessary for aggregating the specialized capabilities of countries within each economic sector \citep{mealyEconomicComplexityGreen2022}. Unlike the original approach in \cite{mealyEconomicComplexityGreen2022}, we slightly modify the equation to accommodate multiple sector groupings. The adjustment corresponds to subtracting a constant reference from the previously grouped PCI values, setting a new zero-mean reference for each economic sector. Additional mathematical details supporting this modification are available in Supplementary Section SB. %\ref{annex:index-validity}.

\begin{equation}\label{eqn:pci-plus}
 PCI_{p}^{s} = PCI_{p\epsilon s} - d(min(PCI_{p\epsilon s}), 0)
\end{equation}

After calculating the SCI values, they are normalized to a range between 0 and 1, as specified in equation \eqref{eqn:sci-norm}, to represent the level of sophistication within sector $s$ for each country $c$. This normalization process also accounts for differences in magnitude resulting from the varying number of products across sectors.

\begin{equation}\label{eqn:sci-norm}
 \widetilde{SCI}_{cs} = \frac{SCI_{cs} - \min_{c}(SCI_{cs})}{\max_{c}(SCI_{cs}) - \min_{c}(SCI_{cs})}
\end{equation}

\subsection{Econometric Model}

Following the derivation of the SCI, this section outlines the econometric model and statistical techniques used to test the EKC hypothesis across the twelve economic sectors empirically. We discuss the formulation of the environment-economy relation, the implementation of statistical robustness checks, and the methodologies used to analyze the relationship between environmental indicators and economic sectors as captured by the SCI.

\subsubsection{Model Specification}
The econometric analysis focuses on CO\textsubscript{2} emissions as the primary environmental indicator, leveraging country-level data due to the scarcity and inconsistency of sector-specific emissions data. The key explanatory variables are the Sectoral Complexity Index ($SCI_{cs}$) for each sector and its squared term ($SCI_{cs}^{2}$), included to test the non-linear sector-level relationship proposed by the EKC hypothesis. We include additional covariates to account for broader macroeconomic and structural factors that influence environmental outcomes across different stages of economic development, as introduced in Section \ref{sec:data-cvars}.

Given the national-level CO\textsubscript{2} emissions data and the interdependence of economic sectors within a country, the regression analysis considers all sectors jointly to account for their combined effects. With yearly data from 1995 to 2020, the data panel consists of 25 years and 24 sectors, severely limiting the sophistication of potential panel data models. Therefore, we adopt a cross-sectional quantile regression approach \citep{waldmannQuantileRegressionShort2018} to address this constraint and account for the heterogeneous distribution of emissions, developmental trajectories, and $SCI_{cs}$ values. This method provides flexibility in the regression slopes across the CO\textsubscript{2} emissions distribution, allowing for a robust analysis of the heterogeneous environment-economy relationship \citep{QRegArgument} irrespective of country-specific effects. Additionally, quantile regression addresses the robustness of the estimation of results. Equation \eqref{eqn:qreg-objective} presents the underlying quantile regression model.

\begin{equation}\label{eqn:qreg-objective}
 \underset{\beta}{\min}\sum_{i:y_{i}\ge x_{i}^{T}\beta}\tau\left| y_{i} - x_{i}^{T}\beta \right| + \sum_{i:y_{i} < x_{i}^{T}\beta}(1-\tau)\left| y_{i} - x_{i}^{T}\beta \right|
\end{equation}

In this formulation, the objective function minimizes the coefficients vector $\beta$, where $\tau$ denotes the target quantile of the dependent variable $y_{i}$, and $i$ represents each data point. The summations capture the absolute differences between observed and predicted values, with residuals weighted by $\tau$ or $1-\tau$ depending on whether the observed values exceed or fall below the predicted quantile.

Equation \eqref{eqn:qreg-model} presents the environment-economy relationship model utilized in this study, for which we compute empirical quantile regressions. CO\textsubscript{2} emissions per capita are used as the environmental degradation indicator, reflecting their relevance and widespread use in recent literature \citep{zhangEnvironmentalKuznetsCurve2019, khezriEnvironmentalImplicationsEconomic2022}. To account for the diverse economic structures of countries at varying stages of development, we categorize the data by income levels as defined by the World Bank: \textit{High income}, \textit{Upper middle income}, \textit{Lower middle income}, and \textit{Low income}. An additional \textit{All income} category encompasses all countries irrespective of income classification.

\begin{equation}\label{eqn:qreg-model}
 Q_{CO_{2}\ Emissions_{log_{gi}}}(\tau | \overrightarrow{SCI}_{gi}, X_{gi}) = a_{0\tau g} + \beta_{1\tau g} \overrightarrow{SCI}_{gi} + \beta_{2\tau g}\overrightarrow{SCI\textsuperscript{2}}_{gi} + \beta_{3\tau g}X_{gi} + \epsilon_{gi\tau}
\end{equation}

In this equation, $\overrightarrow{SCI}_{gi}$ represents the vector of SCI values for all defined economic sectors, $\tau$ represents the corresponding quantile, $i$ refers to the data point, and $g$ indicates the income grouping. $X_{gt}$ is the matrix of covariates, as specified in equation \eqref{eqn:control-vars}, while $\epsilon_{gi\tau}$ captures the unobserved effects unaccounted by the proposed model.

\begin{equation}\label{eqn:control-vars}
 X_{gt} = \left[ 
            \begin{matrix}
 Economic\ Globalisation \\
 Energy\ Consumption\ per\ Capita \\
 Environmental\ Patents\ per\ Capita \\
 Renewable\ energy\ consumption\ (\%\ of\ energy\ consumption) \\
 Total\ natural\ resources\ rents\ (\%\ of\ GDP) \\
 Urban\ population\ (\%\ of\ total\ population)
            \end{matrix}
 \right]
\end{equation}

By incorporating key control variables that significantly influence environmental outcomes, this study aims to provide a robust analysis of the impact of individual economic sectors' sophistication on the economic-environment relationship. \textit{Economic globalisation}, measured using the KOF Globalisation Index \citep{KOFGlobalisationIndex}, captures the degree of trade liberalization and economic interdependence between a country and its international partners. While economic complexity reflects the nature of traded goods and represents a country's embedded knowledge and productive capabilities, economic globalisation offers insights into how these capabilities interact within the global economic network \citep{wangReinvestigatingEnvironmentalKuznets2024}.

\textit{Energy consumption per capita} indicates energy demands associated with economic activities, production processes, and consumption patterns. It also acts as a proxy for technological and industrial development, reflecting the energy intensity required to achieve a given level of economic output. Additionally, \textit{Renewable energy consumption}, expressed as a share of total energy consumption, is included to account for variations in CO\textsubscript{2} emissions that may arise from differences in energy sources. Both energy indicators ensure that overall energy consumption or the environmental impact of energy production does not confound the effects attributed to the SCI.

Similarly, total natural resource rents are included as a control variable to account for the intensity of environmental impacts driven by resource-intensive economic activities. This variable is particularly relevant due to the substantial variation in reliance on natural resource rents across countries at different stages of economic development \citep{alvaradoEcologicalFootprintEconomic2021}. By controlling for total natural resource rents and isolating the effects of natural resource dependence, the analysis allows for a more robust assessment of the SCI's influence on the economic-environment relationship.

In addition, we incorporate the urban population percentage to capture the environmental implications of urbanization. Urbanization often correlates with higher energy consumption, increased infrastructure demands, and shifts in production and consumption patterns, all of which have significant environmental implications \citep{liangEffectUrbanizationEnvironmental2019}. Lastly, we include environmental patents per capita to account for the role of innovation in shaping environmental outcomes \citep{choiLinksEnvironmentalInnovation2018}. These control variables collectively provide a comprehensive framework for analyzing the interaction between sectoral economic complexity and environmental outcomes, ensuring robust and context-sensitive findings.

\subsubsection{Statistical Tests}
We conduct statistical tests before implementing the proposed model to ensure its robustness. The \cite{ShapiroWilk} test and the \cite{ADFTest} test are conducted to evaluate the normality of the SCI distributions. If these tests reject the normality hypothesis, the assumption of heterogeneity is supported, thereby justifying the use of a quantile regression model.

Finally, to prevent spurious regression results, we calculate the \textit{Variance Inflation Factor} (VIF) \citep{VIF} to identify and address potential multicollinearity among the independent variables.


\section{Results and Discussion}

\subsection{Assessing the SCI}

\subsubsection{Economic Sectors Characteristics}
\label{sec:economic-sectors}
Figure 1 illustrates the PCI distributions for all defined economic sectors based on the product classification. The distributions are distinct and consistent across sectors, each covering different PCI ranges and exhibiting cohesive characteristics. The alignment of product examples within representative quantiles further validates the manual classification process and reinforces the idea of shared capabilities among similar products \citep{hidalgoProductSpaceConditions2007}.

The \TE\ sector emerges as the least complex, with a mean PCI of -1.031. This sector is characterized by a wide array of low-complexity products, particularly in knitting and sewing, while the more complex aspects relate to textile manufacturing.

Both the \AG\ and \FI\ sectors also exhibit low complexity, with mean PCI values of -0.870 and -0.889, respectively. The \FO\ sector, with a mean PCI of -0.396, displays a higher complexity, consistent with its reliance on products from the \AG\ and \FI\ sectors combined with additional manufacturing processes that require additional capabilities.

The \MI\ sector shows a narrow range of complexity, with a mean PCI of -0.453, indicating low diversity and low potential sophistication. This sector's complexity is primarily influenced by the availability and quality of ores, as evidenced by the high ubiquity of low-complexity products such as aluminum and copper ores \citep{MiningSector}.

The \OT\ and \WO\ sectors have average complexity, with mean PCI values of -0.075 and 0.118, respectively. The \OT\ sector's lack of distinctive characteristics is consistent with its diverse mix of products not associated with any specific sector. The \WO\ sector displays a broad complexity range, from low-complexity logs to high-complexity coated papers and chemically derived products.

Among the high-complexity sectors, \IR\ has a mean PCI of 0.183, featuring low-complexity products such as wire and scrap and high-complexity products such as stainless steel alloys. The \ME\ sector, with a mean PCI of 0.315, is of higher complexity than \IR, reflecting its wider diversity of products with a higher degree of sophistication \citep{felipeProductComplexityEconomic2012}.

The \PE\ sector, with a mean PCI of 0.548, exhibits greater complexity and sophistication, covering a wide range of products from natural hydrocarbons to highly specialized chemicals. This sector also demonstrates high product diversity, driven by the ease of standardized classification of chemicals.

Finally, the \EL\ and \MA\ sectors are among the most complex, with mean PCI values of 0.536 and 0.792, respectively. Both sectors require substantial knowledge integration across various fields, resulting in high product complexity and a high potential for sophistication \citep{10.1093/icc/dtm006, Turco2020The}.

\subsubsection{Income Levels and CO2 Emissions Across Economic Sectors}
After calculating the SCI for each economic sector, we examined the relationship between SCI values and CO\textsubscript{2} emissions across different income levels. Figure 2 illustrates these relationships on a log scale, with each point representing a country's measure in a given year, categorized by income level. The data reveals a distinct clustering of sectors by income level, with a smooth transition from low-income to high-income countries.

The analysis demonstrates varying degrees of positive correlation between CO\textsubscript{2} emissions and SCI values across sectors. High-complexity sectors, such as \EL, \IR, \MA, \ME, and \PE, exhibit strong positive correlations and some indications of a non-linear relationship. Conversely, lower-complexity sectors, including \AG, \FI, and \TE, display weaker correlations with CO\textsubscript{2} emissions, indicating that the economic sophistication of these sectors less influences emissions.

Furthermore, low-complexity sectors, such as \AG, \FO, and \TE, exhibit broader spreads in high-income groups, suggesting that these sectors are less associated with increased income and therefore do not necessarily achieve higher sophistication in wealthier countries. In contrast, high-complexity sectors show narrower spreads, with SCI values converging at lower CO\textsubscript{2} emissions as sophistication within the sector increases. This divergence highlights the varying roles that different sectors might play in economic development and environmental degradation.

\subsubsection{Countries, Sectors, and SCI}
\label{sec:countries-sci}
Figure 3 presents the relationship between the SCI and CO\textsubscript{2} emissions for the economic sectors in 2020, with each country's data point represented by its national flag. This visualization provides a detailed depiction of the distribution of SCI values across countries and their corresponding CO\textsubscript{2} emissions, offering insight into the sector-specific environmental relationship, the economic structure of individual countries, and how the SCI can capture these nuances.

The scatter plots reveal distinct patterns among sectors, reflecting the characteristics apparent from Figure 3. Furthermore, individual countries exhibit varying degrees of sophistication within the considered sectors. 

Including country flags in the scatter plots enhances the visual clarity of the environment-economic sectors' relationships, allowing for a nuanced understanding of how different countries contribute to CO\textsubscript{2} emissions within each sector regarding their level of economic sophistication. 

Notably, the SCI values of countries in each sector show very consistent behavior with expected economic notions. The \AG\ sector is led by the Netherlands, Spain, and France, which are globally large exporters of agricultural products characterized by providing high-quality, non-ubiquitous goods. Meanwhile, the United States, as the world's largest exporter of agricultural products, ranks seventh due to the high ubiquity of its products, particularly grains and livestock. 

Germany, Japan, the United States, and China lead the \EL\ sector. Interestingly, some countries, such as Malaysia, the Philippines, the Czech Republic, Hungary, Austria, and Mexico, show high sophistication in this sector. In contrast, their SCI in other sectors is relatively low. These countries are known to manufacture and export highly complex electronic products due to intensive research and development, as in Austria, or due to offshore production from developed countries, such as in Malaysia, highlighting how cross-border economic interactions drive local economic development and environmental degradation \citep{balsa-barreiroGlobalizationShiftingCenters2019}. Similarly, Denmark and Norway strongly lead the \FI\ sector. The \IR\ sector is led by Italy, followed by Austria and Germany, while the \MA\ sector is led by Germany and followed by Italy.
On the other hand, the \ME\ sector is led by Germany, Japan, Austria, and Italy in that order. China strongly leads the \OT\ sector and is the most sophisticated country in the \TE\ sector, followed closely by Italy, Turkey, and Pakistan. Lastly, Sweden leads the \WO\ sector.

However, despite similar levels of sectoral complexity, the CO\textsubscript{2} emissions associated with these countries differ, underscoring the complex interplay between economic sophistication and environmental impact. The overall trend indicates increased sectoral complexity generally correlates with higher CO\textsubscript{2} emissions. However, a noticeable transition point is evident in some high-complexity sectors, where further increases in complexity correspond to a stabilization or even reduction in emissions. This pattern is noticeable in sectors such as \IR\ and \MA, where higher levels of complexity seem to mitigate environmental impacts, likely due to the adoption of more efficient technologies, production processes, and cleaner energy sources.

There is also a high degree of agglomeration in the high-complexity sectors, with countries clustering closely together regarding both SCI values and CO\textsubscript{2} emissions. This clustering suggests that, within these sectors, countries with similar levels of economic sophistication tend to exhibit comparable environmental profiles.

Interestingly, oil-exporting countries emerge as outliers in this analysis, particularly in the \PE\ sector. These countries often display high CO\textsubscript{2} emissions relative to their SCI values, indicating that their environmental impact is heavily skewed due to the emission-intensive extraction of oil \citep{IEAOilExtraction}, which is of relatively low complexity, as seen in panel j) of Figure 1, and contributes to their high environmental impact.

This detailed visualization underscores the diverse environmental implications of sectoral economic complexity, offering valuable insights into each country's specific dynamics and particular stages of economic development.


\subsection{Empirical Results}
\subsubsection{Statistical Tests Results}
The statistical tests validate the assumptions underlying the econometric models and ensure the robustness of the results. A comprehensive set of tests was applied to the complete dataset, with 3,302 observations for each variable. Below, we summarize the key findings from these tests.

First, the descriptive statistics for the complete dataset are presented in Table \ref{tab:descriptive-all}, showing average CO\textsubscript{2} emissions per capita of 2.375 metric tons across all countries. Some control variables, \textit{Economic Globalisation}, \textit{Renewable Energy Consumption}, \textit{Total Natural Resources Rents}, and \textit{Urban Population}, are expressed as percentages of total, ranging from 0.0 to 1.0, with mean values of 0.58, 0.31, 0.07, and 0.60, respectively. To account for differences in population size between countries, we express\textit{CO\textsubscript{2} emissions}, \textit{Energy Consumption}, and \textit{Environmental Patents} as per capita values. Log transformation was applied to CO\textsubscript{2} emissions to account for the vast spread of values and disparities between countries. On the other hand, \textit{Energy Consumption} and \textit{Environmental Patents} per capita values are expressed as \textit{per 10,000 people} to avoid numerical errors due to some considerably small numeric values. Detailed descriptive statistics by income grouping are available in Supplementary Table S2.

We performed the Anderson-Darling and Shapiro-Wilk tests to assess the normality of the variables distribution. As summarized in Table \ref{tab:skewdness-test}, both tests indicate significant deviations from normality across all variables and income groupings. Specifically, the Anderson-Darling test, with a critical value of 1.091 at a 1\% significance level, rejects the null hypothesis for all variables. Similarly, the Shapiro-Wilk test rejects the null hypothesis at a 0.1\% significance level for all variables, confirming the non-normality of the data distribution. These results justify using a quantile regression approach, which does not assume normality and is better suited to handle the heterogeneity observed in the dataset.

Lastly, multicollinearity was evaluated using the VIF, as shown in Table \ref{tab:vif-test}. The results reveal some multicollinearity between the \MA\ SCI and \ME\ SCI in the \textit{All income} grouping, with VIF values of 7.42 and 8.04, respectively. This multicollinearity is more pronounced within the \textit{High income} grouping, where the VIF values for \MA\ SCI and \ME\ SCI are 7.99 and 6.73, respectively. While these VIF values indicate some degree of multicollinearity, they remain below the critical thresholds commonly cited in the literature (10 or 15) \citep{dormannCollinearityReviewMethods2013, obrienCautionRegardingRules2007}. This behavior is consistent with the characteristics of developed countries, which tend to have a broad and interconnected set of industrial capabilities \citep{10.1007/978-3-319-96661-8_46}.


\begin{table}
	\centering
	\caption{Descriptive statistics of variables for \textit{All income} grouping.}
      \label{tab:descriptive-all}
      \begin{adjustbox}{width=\textwidth,center}
      \begin{tabular}{lllllllll}
      \toprule
       & \multicolumn{8}{c}{All\ income} \\
       & N\ Observations & Mean & Std.\ Dev. & Min & Median & Max & Kurtosis & Skewness \\
      \midrule
      CO\textsubscript{2}\ emissions\ (metric\ tons\ per\ capita)\textsubscript{log} & 3302 & 0.865 & 1.443 & -3.106 & 1.208 & 3.864 & -0.227 & -0.648 \\
      Agriculture\ SCI & 3302 & 0.169 & 0.14 & 0.0 & 0.134 & 1.0 & 3.585 & 1.582 \\
      Electronics\ \&\ Instruments\ SCI & 3302 & 0.129 & 0.179 & 0.0 & 0.047 & 1.0 & 3.89 & 1.98 \\
      Fishing\ SCI & 3302 & 0.129 & 0.147 & 0.0 & 0.075 & 1.0 & 3.406 & 1.72 \\
      Food\ \&\ Beverages\ SCI & 3302 & 0.199 & 0.173 & 0.0 & 0.157 & 1.0 & 1.709 & 1.302 \\
      Iron\ \&\ Steel\ SCI & 3302 & 0.144 & 0.165 & 0.0 & 0.077 & 1.0 & 1.613 & 1.5 \\
      Machinery\ SCI & 3302 & 0.106 & 0.147 & 0.0 & 0.04 & 1.0 & 5.144 & 2.185 \\
      Metal\ Products\ SCI & 3302 & 0.125 & 0.135 & 0.0 & 0.075 & 1.0 & 2.929 & 1.67 \\
      Mining\ \&\ Quarrying\ SCI & 3302 & 0.241 & 0.194 & 0.0 & 0.193 & 1.0 & 0.386 & 0.973 \\
      Other\ Manufacturing\ SCI & 3302 & 0.097 & 0.119 & 0.0 & 0.051 & 1.0 & 8.22 & 2.335 \\
      Petroleum,\ Chemicals\ \&\ Non-Metals\ SCI & 3302 & 0.133 & 0.158 & 0.0 & 0.076 & 1.0 & 4.975 & 2.127 \\
      Textiles\ \&\ Wearing\ Apparel\ SCI & 3302 & 0.135 & 0.137 & 0.0 & 0.093 & 1.0 & 3.811 & 1.757 \\
      Wood\ \&\ Paper\ SCI & 3302 & 0.171 & 0.172 & 0.0 & 0.11 & 1.0 & 1.598 & 1.388 \\
      Economic\ Globalisation & 3302 & 0.575 & 0.162 & 0.179 & 0.564 & 0.949 & -0.767 & 0.061 \\
      Energy\ Consumption\ per\ Capita & 3302 & 0.228 & 0.262 & 0.011 & 0.131 & 2.143 & 10.199 & 2.708 \\
      Environmental\ Patents\ per\ Capita & 3302 & 0.057 & 0.143 & 0.0 & 0.002 & 1.278 & 16.367 & 3.708 \\
      Renewable\ energy\ consumption\ (\%\ of\ total\ final\ energy\ consumption) & 3302 & 0.309 & 0.288 & 0.0 & 0.215 & 0.967 & -0.704 & 0.771 \\
      Total\ natural\ resources\ rents\ (\%\ of\ GDP) & 3302 & 0.069 & 0.109 & 0.0 & 0.022 & 0.661 & 6.184 & 2.45 \\
      Urban\ population\ (\%\ of\ total\ population) & 3302 & 0.601 & 0.216 & 0.098 & 0.616 & 1.0 & -0.728 & -0.267 \\
      \bottomrule
      \end{tabular}
\end{adjustbox}
\end{table}

\begin{table}
	\centering
	\caption{Skewdness Statistical Tests Results.}
      \label{tab:skewdness-test}
\begin{adjustbox}{width=\textwidth,center}
      \begin{tabular}{lllllllllll}
      \toprule
       & \multicolumn{2}{c}{All\ income} & \multicolumn{2}{c}{High\ income} & \multicolumn{2}{c}{Low\ income} & \multicolumn{2}{c}{Lower\ middle\ income} & \multicolumn{2}{c}{Upper\ middle\ income} \\
       & Anderson-Darling & Shapiro-Wilk & Anderson-Darling & Shapiro-Wilk & Anderson-Darling & Shapiro-Wilk & Anderson-Darling & Shapiro-Wilk & Anderson-Darling & Shapiro-Wilk \\
      \midrule
      Agriculture\ SCI & 96.13 & 0.88(0.0)*** & 21.05 & 0.91(0.0)*** & 15.7 & 0.9(0.0)*** & 23.1 & 0.89(0.0)*** & 8.06 & 0.95(0.0)*** \\
      Electronics\ \&\ Instruments\ SCI & 316.0 & 0.73(0.0)*** & 16.75 & 0.93(0.0)*** & 66.14 & 0.59(0.0)*** & 96.06 & 0.63(0.0)*** & 54.16 & 0.75(0.0)*** \\
      Fishing\ SCI & 185.69 & 0.81(0.0)*** & 30.09 & 0.89(0.0)*** & 37.12 & 0.8(0.0)*** & 57.51 & 0.82(0.0)*** & 25.14 & 0.89(0.0)*** \\
      Food\ \&\ Beverages\ SCI & 96.11 & 0.89(0.0)*** & 11.59 & 0.95(0.0)*** & 19.15 & 0.9(0.0)*** & 7.74 & 0.96(0.0)*** & 4.41 & 0.96(0.0)*** \\
      Iron\ \&\ Steel\ SCI & 231.5 & 0.8(0.0)*** & 21.11 & 0.93(0.0)*** & 67.14 & 0.59(0.0)*** & 81.9 & 0.73(0.0)*** & 33.41 & 0.85(0.0)*** \\
      Machinery\ SCI & 322.15 & 0.71(0.0)*** & 20.45 & 0.92(0.0)*** & 41.73 & 0.75(0.0)*** & 74.77 & 0.76(0.0)*** & 37.07 & 0.83(0.0)*** \\
      Metal\ Products\ SCI & 189.32 & 0.81(0.0)*** & 10.15 & 0.96(0.0)*** & 47.51 & 0.71(0.0)*** & 39.49 & 0.82(0.0)*** & 19.7 & 0.9(0.0)*** \\
      Mining\ \&\ Quarrying\ SCI & 81.72 & 0.91(0.0)*** & 8.05 & 0.97(0.0)*** & 37.25 & 0.77(0.0)*** & 20.11 & 0.91(0.0)*** & 13.54 & 0.92(0.0)*** \\
      Other\ Manufacturing\ SCI & 216.88 & 0.76(0.0)*** & 12.3 & 0.95(0.0)*** & 79.81 & 0.48(0.0)*** & 80.69 & 0.58(0.0)*** & 49.46 & 0.67(0.0)*** \\
      Petroleum,\ Chemicals\ \&\ Non-Metals\ SCI & 240.2 & 0.75(0.0)*** & 31.97 & 0.9(0.0)*** & 92.41 & 0.49(0.0)*** & 62.52 & 0.71(0.0)*** & 11.91 & 0.9(0.0)*** \\
      Textiles\ \&\ Wearing\ Apparel\ SCI & 141.87 & 0.83(0.0)*** & 31.89 & 0.87(0.0)*** & 56.38 & 0.72(0.0)*** & 29.63 & 0.87(0.0)*** & 38.2 & 0.78(0.0)*** \\
      Wood\ \&\ Paper\ SCI & 159.89 & 0.85(0.0)*** & 14.48 & 0.95(0.0)*** & 31.88 & 0.76(0.0)*** & 25.77 & 0.89(0.0)*** & 12.05 & 0.93(0.0)*** \\
      \bottomrule
      \end{tabular}
\end{adjustbox}
{\centering\tiny Note: * p\textless0.05, ** p\textless0.01, *** p\textless0.001\par}
\hfill
\end{table}


\begin{table}
	\centering
	\caption{Variance Inflation Factor Statistical Tests Results.}
      \label{tab:vif-test} 
\begin{adjustbox}{width=\textwidth,center}
      \begin{tabular}{llllll}
      \toprule
      & \multicolumn{5}{c}{VIF} \\
      & All\ income & High\ income & Upper\ middle\ income & Lower\ middle\ income & Low\ income \\
      \midrule
      Agriculture\ SCI & 2.39 & 3.16 & 2.51 & 1.86 & 1.88 \\
      Electronics\ \&\ Instruments\ SCI & 5.63 & 5.23 & 3.82 & 3.21 & 4.73 \\
      Fishing\ SCI & 1.74 & 2.35 & 1.33 & 1.56 & 1.52 \\
      Food\ \&\ Beverages\ SCI & 4.85 & 6.57 & 3.73 & 2.6 & 2.71 \\
      Iron\ \&\ Steel\ SCI & 5.19 & 4.53 & 5.73 & 4.6 & 5.3 \\
      Machinery\ SCI & 7.42 & 7.99 & 5.76 & 3.4 & 4.12 \\
      Metal\ Products\ SCI & 8.04 & 6.73 & 6.19 & 4.76 & 4.74 \\
      Mining\ \&\ Quarrying\ SCI & 3.57 & 3.33 & 3.33 & 3.67 & 3.49 \\
      Other\ Manufacturing\ SCI & 5.24 & 6.29 & 5.8 & 4.1 & 5.45 \\
      Petroleum,\ Chemicals\ \&\ Non-Metals\ SCI & 5.94 & 5.46 & 7.04 & 4.38 & 5.62 \\
      Textiles\ \&\ Wearing\ Apparel\ SCI & 2.8 & 3.42 & 3.44 & 3.45 & 3.03 \\
      Wood\ \&\ Paper\ SCI & 4.19 & 4.16 & 4.34 & 2.79 & 2.12 \\
      Economic\ Globalisation & 2.39 & 1.26 & 1.69 & 1.35 & 1.49 \\
      Energy\ Consumption\ per\ Capita & 2.01 & 2.46 & 1.6 & 2.71 & 2.15 \\
      Environmental\ Patents\ per\ Capita & 2.39 & 2.38 & 1.58 & 1.44 & 1.29 \\
      Renewable\ energy\ consumption\ (\%\ of\ total\ final\ energy\ consumption) & 2.47 & 2.69 & 1.36 & 2.17 & 3.13 \\
      Total\ natural\ resources\ rents\ (\%\ of\ GDP) & 1.68 & 2.72 & 2.11 & 1.57 & 1.37 \\
      Urban\ population\ (\%\ of\ total\ population) & 2.71 & 1.68 & 1.35 & 1.77 & 2.28 \\
      \bottomrule
      \end{tabular}
\end{adjustbox}
\end{table}


\subsubsection{Regression Results}
\paragraph{All Income Levels.}
The regression analysis across all income levels reveals a heterogeneous and statistically significant relationship between SCI and CO\textsubscript{2} emissions, with notable quantile variations. Table \ref{tab:all-income-coeffs} provides the complete coefficient estimates and significance levels calculated by the robust errors method. A detailed discussion of the coefficients' relations is available in Supplementary Section SA.

All control variables exhibit strong significance across quantiles, reinforcing model robustness. Economic Globalization and Energy Consumption per Capita display a consistently positive relationship with emissions, with coefficients increasing from 0.38 at the 0.1 quantile to 0.88 at the 0.9 quantile for globalization and from 1.40 to 2.30 for energy consumption. This pattern aligns with existing literature on the environmental impact of global economic integration and energy consumption \citep{KAIS20161101}. Conversely, \textit{Renewable Energy Consumption} correlates negatively with emissions, with coefficients rising from -3.14 at the 0.1 quantile to -2.04 at the 0.9 quantile, suggesting a diminishing but persistent mitigation effect \citep{Jaforullah2015Does}.

Environmental Patents per Capita exhibit an initially positive but later negative relationship with emissions, shifting from 0.51 at the 0.1 quantile to -0.20 at the 0.9 quantile. This trend aligns with the notion that innovation alone does not immediately translate into emissions reductions, requiring complementary policy measures \citep{TOBELMANN2020118787}. Total Natural Resources Rents consistently increase emissions, peaking at median quantiles before slightly declining, reflecting the sustained environmental costs of resource-dependent economies \citep{wangImpactRiskFactors2024}. Urban Population Share correlates positively with emissions at all quantiles. However, its influence declines at higher quantiles, indicating that urban expansion has potential mitigation through infrastructure and socioeconomic adaptations \citep{land12050981}.


\begin{table}[ht]
	\centering
	\caption{All income Regression Coefficients Results.}
      \label{tab:all-income-coeffs}
\begin{adjustbox}{width=\textwidth,center}
      \begin{tabular}{lllllllllll}
      \toprule
       &  & \multicolumn{9}{c}{All\ income} \\
       & Quantile & 0.1 & 0.2 & 0.3 & 0.4 & 0.5 & 0.6 & 0.7 & 0.8 & 0.9 \\
      Dependent Var & Independent Vars &  &  &  &  &  &  &  &  &  \\
      \midrule
      \multirow[t]{31}{*}{CO\textsubscript{2}\ emissions\ (metric\ tons\ per\ capita)\textsubscript{log}} & Agriculture\ SCI & 0.323 & 0.595*** & 0.693*** & 0.589*** & 0.478** & 0.247 & 0.034 & -0.145 & -0.361* \\
      & Agriculture\ SCI\textsuperscript{2} & -0.517 & -0.523* & -0.684** & -0.716** & -0.668** & -0.53* & -0.397 & -0.263 & -0.128 \\
      & Electronics\ \&\ Instruments\ SCI & -0.432 & -0.744*** & -0.955*** & -0.999*** & -0.956*** & -0.967*** & -0.779*** & -0.584** & -0.477** \\
      & Electronics\ \&\ Instruments\ SCI\textsuperscript{2} & 0.442 & 0.698** & 0.912*** & 0.881*** & 0.836*** & 0.906*** & 0.849*** & 0.582* & 0.365 \\
      & Fishing\ SCI & 0.416* & 0.072 & 0.021 & 0.055 & 0.026 & -0.109 & -0.579*** & -0.849*** & -0.696*** \\
      & Fishing\ SCI\textsuperscript{2} & -0.116 & 0.114 & 0.09 & 0.104 & 0.256 & 0.432 & 1.276*** & 1.528*** & 0.878*** \\
      & Food\ \&\ Beverages\ SCI & -0.86*** & -0.368* & -0.261 & -0.151 & 0.015 & 0.147 & 0.518* & 0.604** & 0.123 \\
      & Food\ \&\ Beverages\ SCI\textsuperscript{2} & 0.251 & -0.335 & -0.256 & -0.211 & -0.311 & -0.346 & -0.53* & -0.55* & 0.041 \\
      & Iron\ \&\ Steel\ SCI & 1.667*** & 1.316*** & 1.162*** & 0.903*** & 0.738*** & 0.743** & 0.608* & 0.62* & 0.423 \\
      & Iron\ \&\ Steel\ SCI\textsuperscript{2} & -2.371*** & -2.014*** & -1.894*** & -1.532*** & -1.223*** & -1.295*** & -1.151** & -0.984** & -0.802* \\
      & Machinery\ SCI & 0.404 & 0.495 & 0.524 & 0.546 & 0.678* & 0.455 & 0.53 & 0.536 & 0.395 \\
      & Machinery\ SCI\textsuperscript{2} & -0.158 & -0.245 & -0.319 & -0.494 & -0.676* & -0.58 & -0.772* & -0.81* & -0.729 \\
      & Metal\ Products\ SCI & 1.848*** & 1.75*** & 1.229*** & 1.028*** & 0.984*** & 0.928** & 1.282*** & 1.476*** & 1.072*** \\
      & Metal\ Products\ SCI\textsuperscript{2} & -2.84*** & -2.311*** & -1.41*** & -0.732 & -0.762 & -0.49 & -1.008 & -1.158* & -0.53 \\
      & Mining\ \&\ Quarrying\ SCI & 0.04 & 0.036 & 0.126 & 0.409** & 0.799*** & 0.71*** & 0.902*** & 1.008*** & 0.78*** \\
      & Mining\ \&\ Quarrying\ SCI\textsuperscript{2} & 0.299 & 0.161 & 0.07 & -0.126 & -0.456* & -0.388* & -0.618** & -0.705*** & -0.431* \\
      & Other\ Manufacturing\ SCI & 0.701* & 1.011*** & 1.02*** & 0.785* & 0.513 & 0.438 & 0.137 & 0.502 & 0.944*** \\
      & Other\ Manufacturing\ SCI\textsuperscript{2} & 0.096 & -0.058 & -0.24 & -0.018 & 0.284 & 0.266 & 0.446 & -0.056 & -0.519 \\
      & Petroleum,\ Chemicals\ \&\ Non-Metals\ SCI & -1.292*** & -1.546*** & -1.047*** & -0.882*** & -0.782** & -0.639* & -0.836** & -1.024*** & -0.546 \\
      & Petroleum,\ Chemicals\ \&\ Non-Metals\ SCI\textsuperscript{2} & 0.434 & 0.542 & -0.098 & -0.248 & -0.244 & -0.315 & -0.108 & 0.171 & -0.187 \\
      & Textiles\ \&\ Wearing\ Apparel\ SCI & -0.371 & -0.614* & -0.823*** & -0.544* & -0.412* & -0.381 & -0.423* & -1.05*** & -1.125*** \\
      & Textiles\ \&\ Wearing\ Apparel\ SCI\textsuperscript{2} & 0.639 & 0.812 & 1.182** & 0.865* & 0.508 & 0.488 & 0.695* & 1.478*** & 1.484*** \\
      & Wood\ \&\ Paper\ SCI & 2.756*** & 2.915*** & 2.724*** & 2.108*** & 1.66*** & 1.709*** & 1.41*** & 0.922*** & 0.989*** \\
      & Wood\ \&\ Paper\ SCI\textsuperscript{2} & -2.314*** & -2.504*** & -2.297*** & -1.654*** & -1.286*** & -1.296*** & -1.279*** & -1.063*** & -1.293*** \\
      & Economic\ Globalisation & 0.384*** & 0.599*** & 0.709*** & 0.681*** & 0.716*** & 0.74*** & 0.874*** & 0.826*** & 0.875*** \\
      & Energy\ Consumption\ per\ Capita & 1.399*** & 1.308*** & 1.339*** & 1.375*** & 1.331*** & 1.371*** & 1.49*** & 1.823*** & 2.296*** \\
      & Environmental\ Patents\ per\ Capita & 0.51*** & 0.486*** & 0.485*** & 0.434*** & 0.41*** & 0.428*** & 0.464*** & 0.133* & -0.201*** \\
      & Intercept & -0.4*** & -0.372*** & -0.244*** & -0.103 & -0.02 & 0.186*** & 0.225*** & 0.313*** & 0.314*** \\
      & Renewable\ energy\ consumption\ (\%\ of\ total\ final\ energy\ consumption) & -3.142*** & -3.116*** & -3.01*** & -2.91*** & -2.852*** & -2.845*** & -2.72*** & -2.345*** & -2.044*** \\
      & Total\ natural\ resources\ rents\ (\%\ of\ GDP) & 0.955*** & 1.057*** & 0.96*** & 1.062*** & 1.226*** & 1.187*** & 1.114*** & 0.882*** & 0.916*** \\
      & Urban\ population\ (\%\ of\ total\ population) & 1.09*** & 1.141*** & 1.003*** & 0.957*** & 0.885*** & 0.71*** & 0.637*** & 0.678*** & 0.748*** \\
      \cline{1-11}
      \bottomrule
      \end{tabular}
      \end{adjustbox}
      {\centering\tiny Note: * p\textless0.05, ** p\textless0.01, *** p\textless0.001\par}
      \hfill
\end{table}


The sectoral complexity-emissions relationship exhibits notable differences across economic sectors. The \AG, \IR, \MI, \ME, and \WO\ sectors support the EKC hypothesis, showing positive linear and negative quadratic coefficients, indicating that emissions initially rise with complexity and then decline with further sophistication. For \IR, SCI coefficients decrease from 1.67 at the 0.1 quantile to 0.42 at the 0.9 quantile. In contrast, SCI\textsuperscript{2} coefficients increase from -2.37 to -0.80, suggesting that advancements such as adopting more efficient electric arc furnaces, cleaner energy sources, and economies of scale contribute to emissions reduction \citep{Wang2017Analysis}. The \MI\ sector follows a similar trajectory, with a peak 1.01 SCI coefficient at the 0.8 quantile, reinforcing its emissions-intensive nature at moderate complexity levels before mitigation effects emerge at higher sophistication with 0.90 SCI\textsuperscript{2} coefficient at the corresponding quantile. A similar trend is apparent in \ME, where emissions increase with complexity, peaking at 1.85 at the 0.1 quantile before declining to 1.07 at the 0.9 quantile. On the contrary, the SCI\textsuperscript{2} coefficients peak at -2.84 at the 0.1 quantile, increasing to -1.41 at the 0.3 quantile. The observed relationship aligns with findings in metal energy-intensive industries such as aluminum and nickel production, where efficiency improvements and cleaner processing technologies reduce emissions over time \citep{Strezov2021Life}. Notably, the \ME\ sector's SCI\textsuperscript{2} coefficients are insignificant at the median and higher emission quantiles, showing that, on the one hand, sophistication enables emission reductions. However, on the other hand, they boost demand for complex and emission-intensive metals, creating further increases in emissions \citep{MetalsProduction}.

The \WO\ sector exhibits the strongest EKC behavior across all quantiles, with a peak 2.92 SCI coefficient at lower quantiles, declining at higher quantiles to 0.99. Conversely, the SCI\textsuperscript{2} coefficients peak at -2.50 at the 0.2 quantile before increasing to -1.29 at the 0.9 quantile. This sector transitions from land- and emissions-intensive industrial processes to more sustainable and scalable industrial practices as complexity increases, reinforcing the role of processes and technological upgrades in mitigating emissions. The \AG\ sector also supports an EKC pattern in lower emission quantiles reflecting emissions reductions via technological and efficiency improvements in early agriculture development \citep{AgricultureEnvironment}.

In contrast, the \EL\ sector exhibits a U-shaped relationship, contradicting the EKC hypothesis. SCI coefficients decrease from -0.74 at the 0.2 quantile to -1.0 at the 0.4 quantile before rising again to -0.48 at the 0.9 quantile. In contrast, the SCI\textsuperscript{2} coefficients increase from 0.70 at the 0.2 quantile, peaking at 0.91 at the 0.3 quantile and declining to 0.58 at the 0.8 quantile. This trend suggests that emissions reductions in electronics occur at early sophistication stages but reverse at high sophistication due to the increased complexity of manufacturing processes and the offshore production of lower-complexity products to developing countries \citep{ElectronicsOffshore}, as mentioned in Section \ref{sec:countries-sci}. The \FI\ and \TE\ sectors also demonstrate U-shaped behavior at high quantiles, with SCI coefficients becoming increasingly negative before rising again. Higher complexity in the \FI\ sector is associated with fuel-intensive activities \citep{parkerFuelUseGreenhouse2018, salaEnergyAuditCarbon2022} and higher complexity in the \TE\ sector is associated with energy-intensive manufacturing that is fueled by cheap emission-intensive energy sources \citep{textilesenvironmental}.

Finally, the \PE\ sector exhibits a consistent negative linear relationship with emissions, with SCI coefficients ranging from -1.55 at the 0.2 quantile to -0.55 at the 0.9 quantile. This pattern suggests that increasing sectoral complexity in \PE\ does lead to emissions reductions, emphasizing the sizeable environmental burden of low-complexity fossil fuel extraction and production and the potential reduction of emissions with higher complexity chemicals production.

The empirical results highlight a nuanced relationship between sectoral economic complexity and CO\textsubscript{2} emissions, revealing heterogeneous patterns across sectors and income levels. The presence of EKC-like behavior in sectors such as \IR, \MI, \ME, and \WO\ underscores the role of technological advancement in mitigating emissions at higher complexity levels. In contrast, the U-shaped relationships observed in the \EL, \FI, and \TE\ sectors suggest that rising complexity does not uniformly lead to lower emissions, often due to structural constraints, energy-intensive production, or globalized supply chain effects. The consistently negative association between SCI and emissions in the \PE\ sector further reinforces the notion that higher complexity can drive efficiency gains and cleaner production methods, particularly in industries with well-defined technological pathways for sustainability transitions. Furthermore, the effects captured by the introduced SCI are statistically significant, and the captured effects' magnitudes are comparable with the effect of selected control variables.

From a policy perspective, these findings challenge the assumption that economic sophistication guarantees environmental improvements. While complexity-driven technological advancements contribute to emissions reductions in select sectors, the emissions trajectory varies significantly based on production processes, resource dependence, and global trade dynamics. These complexities require a sector-sensitive policy framework that accounts for industry-specific emissions behavior rather than applying general regulatory approaches. Moreover, the observed U-shaped emission patterns in \EL\ and \TE\ highlight the risks associated with outsourcing pollution-intensive production to developing countries. Addressing this issue requires international cooperation on environmental standards, sustainable supply chain management incentives, and mechanisms to prevent carbon leakage in globally integrated industries. These findings suggest that while economic sophistication plays a crucial role in shaping emissions trajectories, its impact is neither uniform nor automatic. A more granular, sector-specific approach to sustainability policymaking is essential to align economic complexity with long-term environmental goals.


\begin{table}[ht]
	\centering
	\caption{High income and Upper middle income Regression Coefficients Results.}
\begin{adjustbox}{width=\textwidth,center}
      \begin{tabular}{lllllllllll}
      \toprule
       &  & \multicolumn{9}{c}{High\ income} \\
       & Quantile & 0.1 & 0.2 & 0.3 & 0.4 & 0.5 & 0.6 & 0.7 & 0.8 & 0.9 \\
      Dependent Var & Independent Vars &  &  &  &  &  &  &  &  &  \\
      \midrule
      \multirow[t]{31}{*}{CO\textsubscript{2}\ emissions\ (metric\ tons\ per\ capita)\textsubscript{log}} & Agriculture\ SCI & -0.019 & 0.177 & 0.078 & 0.245 & 0.225 & 0.176 & 0.147 & 0.168 & 0.173 \\
      & Agriculture\ SCI\textsuperscript{2} & 0.13 & -0.156 & -0.036 & -0.256 & -0.252 & -0.224 & -0.219 & -0.309* & -0.334* \\
      & Electronics\ \&\ Instruments\ SCI & -0.397 & -0.402* & -0.378* & -0.106 & 0.086 & 0.228 & 0.413** & 0.545*** & 0.648*** \\
      & Electronics\ \&\ Instruments\ SCI\textsuperscript{2} & 0.674** & 0.472* & 0.411* & 0.185 & 0.025 & -0.141 & -0.266 & -0.36* & -0.386** \\
      & Fishing\ SCI & -0.849*** & -0.725*** & -0.382** & -0.334** & -0.397*** & -0.538*** & -0.443*** & -0.208* & 0.074 \\
      & Fishing\ SCI\textsuperscript{2} & 1.495*** & 1.11*** & 0.736*** & 0.689*** & 0.724*** & 0.985*** & 0.907*** & 0.501*** & -0.042 \\
      & Food\ \&\ Beverages\ SCI & -1.77*** & -0.897*** & -0.535** & -0.412* & -0.104 & 0.066 & 0.242 & 0.315 & 0.524** \\
      & Food\ \&\ Beverages\ SCI\textsuperscript{2} & 1.366*** & 0.583** & 0.315 & 0.252 & 0.014 & -0.114 & -0.332 & -0.326 & -0.532** \\
      & Iron\ \&\ Steel\ SCI & -1.242*** & -1.155*** & -0.921*** & -0.557** & -0.66*** & -0.738*** & -0.533** & -0.515** & -0.534*** \\
      & Iron\ \&\ Steel\ SCI\textsuperscript{2} & 0.782*** & 0.742** & 0.559* & 0.239 & 0.413 & 0.493* & 0.327 & 0.366 & 0.365* \\
      & Machinery\ SCI & 1.086*** & 0.545 & 0.2 & 0.033 & 0.135 & 0.349 & 0.369 & 0.312 & 0.189 \\
      & Machinery\ SCI\textsuperscript{2} & -0.349 & -0.106 & 0.114 & 0.171 & -0.002 & -0.254 & -0.328 & -0.306 & -0.179 \\
      & Metal\ Products\ SCI & 0.57** & 0.662** & 0.402 & 0.241 & 0.43 & 0.455* & 0.378 & 0.294 & 0.088 \\
      & Metal\ Products\ SCI\textsuperscript{2} & -0.817*** & -0.67* & -0.373 & -0.214 & -0.283 & -0.313 & -0.304 & -0.217 & -0.14 \\
      & Mining\ \&\ Quarrying\ SCI & 2.585*** & 2.252*** & 2.074*** & 1.876*** & 1.9*** & 1.89*** & 1.777*** & 1.569*** & 1.349*** \\
      & Mining\ \&\ Quarrying\ SCI\textsuperscript{2} & -1.839*** & -1.609*** & -1.544*** & -1.38*** & -1.407*** & -1.443*** & -1.361*** & -1.231*** & -1.115*** \\
      & Other\ Manufacturing\ SCI & 0.02 & 0.038 & -0.222 & -0.3 & -0.37 & -0.262 & -0.716* & -0.705* & -0.362 \\
      & Other\ Manufacturing\ SCI\textsuperscript{2} & -1.389 & -0.966 & -0.445 & -0.031 & -0.055 & -0.066 & 0.825 & 1.022* & 0.314 \\
      & Petroleum,\ Chemicals\ \&\ Non-Metals\ SCI & -2.024*** & -1.573*** & -1.024*** & -0.929*** & -1.24*** & -1.316*** & -1.166*** & -1.068*** & -1.229*** \\
      & Petroleum,\ Chemicals\ \&\ Non-Metals\ SCI\textsuperscript{2} & 1.518*** & 0.999*** & 0.611** & 0.491* & 0.737*** & 0.859*** & 0.753*** & 0.634*** & 0.772*** \\
      & Textiles\ \&\ Wearing\ Apparel\ SCI & -0.576 & -0.945*** & -1.081*** & -1.192*** & -0.847*** & -0.612** & -0.501** & -0.611*** & -0.166 \\
      & Textiles\ \&\ Wearing\ Apparel\ SCI\textsuperscript{2} & 0.883 & 1.127* & 1.284*** & 1.27*** & 0.703* & 0.391 & 0.188 & 0.313 & -0.117 \\
      & Wood\ \&\ Paper\ SCI & 1.414*** & 1.68*** & 1.195*** & 0.993*** & 0.664*** & 0.141 & -0.012 & -0.127 & 0.073 \\
      & Wood\ \&\ Paper\ SCI\textsuperscript{2} & -0.84*** & -1.126*** & -0.771*** & -0.633*** & -0.294 & 0.17 & 0.223 & 0.228 & -0.028 \\
      & Economic\ Globalisation & 0.076 & -0.154 & -0.231** & -0.306*** & -0.359*** & -0.321*** & -0.285*** & -0.218** & -0.163* \\
      & Energy\ Consumption\ per\ Capita & 0.814*** & 0.894*** & 1.039*** & 1.075*** & 1.112*** & 1.248*** & 1.271*** & 1.357*** & 1.349*** \\
      & Environmental\ Patents\ per\ Capita & -0.019 & 0.238*** & 0.404*** & 0.349*** & 0.352*** & 0.305*** & 0.232*** & 0.163*** & 0.176*** \\
      & Intercept & 1.273*** & 1.337*** & 1.768*** & 1.951*** & 2.0*** & 2.009*** & 2.023*** & 1.997*** & 1.876*** \\
      & Renewable\ energy\ consumption\ (\%\ of\ total\ final\ energy\ consumption) & -2.306*** & -2.351*** & -2.297*** & -2.305*** & -2.239*** & -2.054*** & -1.921*** & -1.706*** & -1.443*** \\
      & Total\ natural\ resources\ rents\ (\%\ of\ GDP) & 0.705*** & 0.736*** & 0.513*** & 0.407*** & 0.514*** & 0.479*** & 0.506*** & 0.376*** & 0.444*** \\
      & Urban\ population\ (\%\ of\ total\ population) & 0.608*** & 0.633*** & 0.157* & 0.038 & -0.004 & -0.067 & -0.11 & -0.11 & -0.018 \\
	\toprule
      &  & \multicolumn{9}{c}{Upper\ middle\ income} \\
      & Quantile & 0.1 & 0.2 & 0.3 & 0.4 & 0.5 & 0.6 & 0.7 & 0.8 & 0.9 \\
      Dependent Var & Independent Vars &  &  &  &  &  &  &  &  &  \\
      \midrule
      \multirow[t]{31}{*}{CO\textsubscript{2}\ emissions\ (metric\ tons\ per\ capita)\textsubscript{log}} & Agriculture\ SCI & 0.837** & 0.556* & 0.185 & -0.631* & -0.842** & -0.802** & -0.965*** & -1.031*** & -1.09*** \\
      & Agriculture\ SCI\textsuperscript{2} & -1.656** & -1.065* & -0.585 & 0.784 & 1.068* & 1.06* & 1.374** & 1.693** & 1.866*** \\
      & Electronics\ \&\ Instruments\ SCI & 0.132 & -0.008 & -0.094 & -0.206 & -0.29 & -0.534* & -0.536* & -0.781*** & -1.546*** \\
      & Electronics\ \&\ Instruments\ SCI\textsuperscript{2} & 0.005 & 0.041 & 0.045 & 0.384 & 0.485 & 0.66* & 0.582* & 0.806** & 1.798*** \\
      & Fishing\ SCI & 0.519 & 0.143 & -0.054 & -0.344 & -0.844** & -0.576* & -0.351 & -0.469 & -0.46 \\
      & Fishing\ SCI\textsuperscript{2} & -0.564 & -0.162 & 0.113 & 0.718 & 2.449*** & 1.628* & 1.154 & 1.303 & 0.736 \\
      & Food\ \&\ Beverages\ SCI & -1.696*** & -0.679* & -0.569* & -0.226 & 0.128 & 0.116 & 0.12 & -0.001 & -0.141 \\
      & Food\ \&\ Beverages\ SCI\textsuperscript{2} & 2.309*** & 1.207* & 1.206** & 1.05* & 0.58 & 0.444 & 0.189 & 0.197 & 0.184 \\
      & Iron\ \&\ Steel\ SCI & 1.454*** & 1.49*** & 1.587*** & 1.223*** & 1.202*** & 1.053*** & 1.445*** & 1.465*** & 1.508*** \\
      & Iron\ \&\ Steel\ SCI\textsuperscript{2} & -2.008*** & -1.838*** & -2.007*** & -1.796*** & -1.606** & -1.341* & -1.799*** & -1.927** & -1.953* \\
      & Machinery\ SCI & 1.768*** & 1.527*** & 1.697*** & 1.856*** & 2.137*** & 2.071*** & 1.927*** & 1.939*** & 2.146*** \\
      & Machinery\ SCI\textsuperscript{2} & -1.525 & -1.798* & -2.322** & -2.133* & -3.145*** & -3.371*** & -3.548*** & -3.89*** & -4.491*** \\
      & Metal\ Products\ SCI & 1.286 & 1.019* & 1.342** & 1.754*** & 1.787*** & 1.363*** & 1.036** & 0.474 & 0.954** \\
      & Metal\ Products\ SCI\textsuperscript{2} & -0.602 & -0.343 & -0.742 & -2.079* & -2.377* & -1.55 & -1.156 & -0.125 & -1.369 \\
      & Mining\ \&\ Quarrying\ SCI & -0.338 & -0.002 & -0.22 & -0.25 & -0.127 & 0.129 & 0.234 & 0.502** & 0.63*** \\
      & Mining\ \&\ Quarrying\ SCI\textsuperscript{2} & 0.664* & 0.254 & 0.478* & 0.706*** & 0.603** & 0.299 & 0.148 & -0.175 & -0.338 \\
      & Other\ Manufacturing\ SCI & -0.274 & 0.025 & -0.034 & -0.332 & -0.283 & 0.275 & 0.228 & 0.517 & 0.924** \\
      & Other\ Manufacturing\ SCI\textsuperscript{2} & 0.608 & 0.491 & 0.512 & 0.729 & 0.632 & 0.019 & 0.124 & -0.29 & -0.942 \\
      & Petroleum,\ Chemicals\ \&\ Non-Metals\ SCI & -0.696 & -0.237 & -0.165 & -1.379** & -2.094*** & -2.093*** & -1.981*** & -1.904*** & -2.025*** \\
      & Petroleum,\ Chemicals\ \&\ Non-Metals\ SCI\textsuperscript{2} & 0.031 & -0.414 & -0.212 & 1.698 & 3.144** & 3.268** & 3.057** & 3.472*** & 4.508*** \\
      & Textiles\ \&\ Wearing\ Apparel\ SCI & -0.274 & 0.093 & -0.018 & 0.232 & 0.079 & -0.296 & -0.506* & -0.534* & -0.76*** \\
      & Textiles\ \&\ Wearing\ Apparel\ SCI\textsuperscript{2} & -0.06 & -0.667* & -0.421 & -0.554 & -0.346 & 0.078 & 0.387 & 0.386 & 0.678* \\
      & Wood\ \&\ Paper\ SCI & 1.422*** & 0.554* & 0.066 & -0.081 & -0.164 & 0.146 & -0.164 & -0.039 & 0.147 \\
      & Wood\ \&\ Paper\ SCI\textsuperscript{2} & -1.724** & -0.468 & 0.175 & 0.114 & 0.046 & -0.319 & 0.642 & 0.641 & 0.328 \\
      & Economic\ Globalisation & 0.226* & -0.049 & 0.036 & 0.036 & 0.151 & 0.145 & 0.218** & 0.291*** & 0.36*** \\
      & Energy\ Consumption\ per\ Capita & 1.702*** & 1.957*** & 2.111*** & 2.61*** & 3.023*** & 3.115*** & 3.318*** & 3.418*** & 3.653*** \\
      & Environmental\ Patents\ per\ Capita & 1.749 & 1.953 & 1.238 & 0.784 & 0.256 & 0.36 & -0.245 & -0.152 & -0.302 \\
      & Intercept & 0.377*** & 0.586*** & 0.558*** & 0.638*** & 0.471*** & 0.568*** & 0.602*** & 0.642*** & 0.552*** \\
      & Renewable\ energy\ consumption\ (\%\ of\ total\ final\ energy\ consumption) & -1.842*** & -1.695*** & -1.474*** & -1.35*** & -1.11*** & -1.064*** & -1.029*** & -0.892*** & -0.403*** \\
      & Total\ natural\ resources\ rents\ (\%\ of\ GDP) & 1.004*** & 1.364*** & 1.442*** & 1.104*** & 1.07*** & 1.194*** & 1.107*** & 1.194*** & 1.359*** \\
      & Urban\ population\ (\%\ of\ total\ population) & 0.25* & 0.037 & 0.108 & 0.192* & 0.338*** & 0.217** & 0.167* & 0.083 & 0.099 \\
            \cline{1-11}
      \bottomrule
      \label{tab:high-um-coefficients}
      \end{tabular}
      \end{adjustbox}
      {\centering\tiny Note: * p\textless0.05, ** p\textless0.01, *** p\textless0.001\par}
      \hfill
\end{table}

\paragraph{By Income Levels.}
Tables \ref{tab:high-um-coefficients} and \ref{tab:lm-li-coefficients} present the coefficient estimates and significance levels for each income group, calculated using robust standard errors. Overall, the results reveal distinct patterns that deepen our understanding of how sectoral complexity and control variables shape CO\textsubscript{2} emissions at varying stages of economic development.

The control variables retain significance across income levels but vary in trend and magnitude. \textit{Economic Globalisation} demonstrates a strongly positive relationship in low- and lower-middle-income countries, where median quantile coefficients decline from 1.77 to 0.59, respectively. The relationship becomes insignificant at the upper-middle-income level before turning significantly negative in high-income levels, with a -0.36 coefficient at the median quantile. This shift suggests that while global integration spurs emission-intensive industrial expansion at early stages, it eventually supports cleaner production in advanced economies \citep{EconomicGlobalisationCO2}. \textit{Energy Consumption per Capita} remains a primary emission driver across all income levels. Its median-quantile coefficient peaks in lower-middle-income countries at 6.68, reflecting intensive industrialization, before declining to 3.02 in upper-middle and 1.11 in high-income levels. \textit{Renewable Energy Consumption} consistently reduces emissions, with the peak negative effects observed in high- and low-income groups at -2.59 and -2.40 at the median quantiles, respectively. \textit{Environmental Patents per Capita} emerge as a significant factor only in high-income economies, corroborating previous observations that technological innovations yield meaningful emissions reductions primarily when supported by robust regulatory frameworks and market conditions \citep{HE2021148908}. \textit{Total Natural Resources Rents} positively influence emissions across all income groups. However, the magnitude varies with coefficient magnitudes of 1.07, 0.51, 0.65, and 0.42 at the median quantiles for the low-, lower-middle-, upper-middle- and high-income groupings. Finally, \textit{Urban Population} is significant in low- and upper-middle-income levels, with the most substantial effect at the low-income level with a magnitude of 2.59 at the median quantile. These results indicate clear transition points where urban population changes are relevant \citep{LI20151107}.

The sectoral analysis reveals nuanced relations between sectoral complexity and CO\textsubscript{2}. In high-income countries, the \IR\ sector exhibits significant negative linear coefficients across all quantiles, peaking at -1.24 at the 0.1 quantile before increasing to -0.53 at the 0.9 quantile. This pattern indicates a consistent decline in CO\textsubscript{2} emissions as SCI increases in high-end sophistication, further supporting the EKC hypothesis within this sector. The findings suggest that technological advancements and cleaner production processes in high-end \IR\ industries contribute to long-term emissions reductions. The \MI\ sector demonstrates a strong and consistent EKC-like relationship, with positive SCI and negative SCI\textsuperscript{2} coefficients across all quantiles. The linear coefficients decrease with increasing quantiles, from 2.59 at the 0.1 quantile to 1.35 at the 0.9 quantile for the SCI. In contrast, the quadratic coefficients increase from -1.84 at the 0.1 quantile to -1.12 at the 0.9 quantile for the SCI\textsuperscript{2}, suggesting that early stage sophistication elevates emissions. However, advanced mining practices become cleaner with increased sophistication \citep{SHAO2016220}, albeit with diminishing returns. Likewise, the \WO\ sector also follows a strong EKC pattern, with significant positive SCI and negative SCI\textsuperscript{2} between the 0.1 and 0.4 quantiles. As with previous sectors, the \WO\ effect magnitude peaks at the 0.1 quantile with 1.41 SCI and -0.84 SCI\textsuperscript{2}, before declining to 0.99 and -0.63, respectively, at the 0.4 quantile.

In contrast, the \PE\ sector exhibits a U-shaped relation, characterized by negative linear coefficients and positive quadratic terms, contravening the EKC hypothesis. The SCI coefficients begin at -2.02 at the 0.1 quantile, increase to 0.93 at the 0.4 quantile, and decline to -1.23 at the 0.9 quantile. A similar pattern is observed in the SCI\textsuperscript{2} coefficients, which start at 1.52 at the 0.1 quantile, decrease to 0.49 at the 0.4 quantile, and then rise again to 0.77 at the 0.9 quantile. This consistent U-shaped pattern aligns with the findings of \cite{moutinhoDeterminantsEnvironmentalKuznets2020}, suggesting that oil-exporting countries within this income category disproportionately influence the results due to their high emission levels despite relatively low sectoral complexity. Such outlier behavior underscores the need for further research that accounts for country-specific characteristics, as this may yield additional insights into the mechanisms underlying emissions trends in resource-dependent economies. Similarly, the \TE\ and \FI\ sectors also follow U-shaped patterns, indicating that while early-stage sectoral sophistication leads to reduced emissions through efficiency gains, emissions intensity stabilizes or reverses at higher complexity levels. This reversal may stem from the increasing energy demands of more advanced production processes or the expansion of complex global supply chains, which introduce additional emissions burdens \citep{10.3389/fenvs.2022.973102}. 

In upper-middle-income countries, the \IR\ and \MA\ sectors exhibit significant inverted U-shaped relationships across all quantiles, supporting a strong EKC transition at this income level driven by these sectors. The SCI and SCI\textsuperscript{2} coefficients of the \IR\ sector are 1.20 and -1.60 at the median quantile, while the \MA\ coefficients are 2.14 and -3.15 at the median quantile. These results suggest that the development and sophistication of the \IR\ and \MA\ sectors are key in the transition to developed economies with a lower environmental footprint, which could be attributed to their energy-intensive nature and the key role that these sectors play in enabling activities in higher-complexity sectors \citep{LIU201864, gonzalez2011iron, Zhi-jun2011Machinery}. In contrast, the \ME\ sector exhibits a strong positive linear relationship, suggesting that increases in sophistication are associated with increases in emissions with a corresponding SCI coefficient of 1.79 at the median quantile. 

The \AG\ sector follows an EKC pattern at low quantiles, with a peak SCI of 0.84 and -1.66 SCI\textsuperscript{2} at the 0.1 quantile. This pattern suggests that initial emission increments are mitigated by further sector sophistication in these quantiles. However, this relationship transitions into a U-shaped pattern at the median and higher emission quantiles, indicating that further increases in sectoral complexity may counteract previous emissions reductions as economic activity intensifies. This shift suggests that, at this stage of economic development, increased agricultural sophistication does not uniformly lead to lower emissions, possibly due to industrialized farming practices and mechanization that contribute to higher energy consumption and environmental impact. Similarly, the \EL\ sector exhibits a U-shaped relationship at high emission quantiles, with SCI and SCI\textsuperscript{2} coefficients of -1.55 and 1.80, respectively, at the 0.9 quantile. This pattern suggests that while early-stage complexity initially reduces emissions—potentially through efficiency gains and process optimization—further sophistication leads to increased emissions, likely driven by the high energy demands of advanced electronics manufacturing and the expansion of complex global supply chains. The \PE\ sector also follows a U-shaped relationship but at median and high emission quantiles, with peak SCI and SCI\textsuperscript{2} coefficients of -2.03 and 4.51 at the 0.9 quantile. Conversely, the \FO\ sector exhibits a U-shaped relationship in low emission quantiles.

In lower-middle-income countries, the \OT\ sector exhibits a strong positive linear correlation with emissions, with the SCI coefficient peaking at 1.85 at the 0.8 quantile. This trend reflects the environmental costs associated with industrial expansion at the early stages of economic development, where increasing complexity corresponds to higher energy consumption and emission-intensive processes. The \AG\ sector follows an EKC pattern at lower quantiles, with SCI and SCI\textsuperscript{2} coefficients reaching 1.41 and -2.2, respectively, at the 0.1 quantile. However, this relationship shifts to a U-shaped pattern at the 0.9 quantile, mirroring the transition observed in upper-middle-income countries. This shift suggests that while early sectoral sophistication initially leads to emissions reductions—potentially through efficiency improvements and modernization—further economic development counteracts these gains as mechanization, input-intensive agricultural practices, and expanded production contribute to higher emissions. Similarly, the \IR\ sector exhibits a U-shaped relationship at lower quantiles, transitioning into an EKC pattern at the 0.9 quantile. The corresponding SCI and SCI\textsuperscript{2} coefficients begin at -1.89 and 1.91 at the 0.1 quantile, respectively, and shift to 0.99 and -1.24 at the 0.9 quantile. This pattern suggests that at the early stages of industrialization increased sectoral complexity does not immediately translate into emissions reductions due to the energy-intensive nature of production processes.

Low-income countries exhibit strong early-stage industrialization effects, where sectors with low complexity primarily drive emissions. The \OT\ sector follows a pronounced EKC pattern, with SCI coefficients increasing significantly from the median to higher quantiles. At the 0.9 quantile, the SCI and SCI\textsuperscript{2} coefficients peak at 6.33 and -4.91, respectively, making the \OT\ sector the predominant driver of emissions at this income level. This trend reflects an initial phase of industrialization characterized by low sectoral complexity and high emissions intensity. Similarly, the \WO\ sector exhibits a strong EKC relationship, with peak SCI and SCI\textsuperscript{2} coefficients of 2.65 and -4.73, respectively, at the 0.6 quantile. This pattern suggests that while early industrial expansion in the wood and paper industry contributes to emissions growth, increasing sectoral sophistication leads to subsequent reductions, likely due to improved production efficiency and shifts toward sustainable practices with decreased deforestation \citep{GRIFFIN2018152}.

In contrast, the \MA\ sector demonstrates a strong negative linear correlation with emissions at higher quantiles, with the SCI coefficient reaching -8.22 at the 0.7 quantile. This result suggests that increased sophistication in the \MA\ sector enables significant efficiency gains, leading to substantial emissions reductions. Other significant relationships are observed in lower-complexity sectors, including \FI, \FO, and \MI, all exhibiting U-shaped patterns across different quantile spans. These findings indicate that while sectoral complexity initially contributes to emissions reductions in these industries, further advancements may reverse this trend due to increased production intensity, resource dependence, or limitations in technological adaptation at low-income levels. 

These results provide robust evidence that the relationship between sectoral economic complexity and CO\textsubscript{2} emissions is highly heterogeneous across income levels. While high-income economies exhibit EKC-like behavior in sectors such as \IR, \MI, and \WO—where increasing complexity leads to long-term emissions reductions—lower-income economies display mixed patterns, with U-shaped relationships emerging in key sectors such as \PE, \TE, and \FI. These results suggest that the environmental benefits of economic complexity are contingent on the stage of economic development and sectoral dynamics.

From a policy perspective, these findings emphasize the importance of tailored interventions that align sectoral development with sustainable growth objectives. High-income economies should prioritize investment in research and development, industrial upgrading, and regulatory frameworks reinforcing cleaner production processes. For upper-middle-income economies, facilitating sectoral transitions by promoting innovation, enforcing environmental standards, and supporting sustainable supply chains will ensure that complexity-driven growth does not prolong environmental degradation.

The results highlight the need for policies that balance industrial expansion with emissions mitigation for lower-middle- and low-income economies. The strong positive relationship between sectoral complexity and emissions in these economies suggests that strategic investments in energy efficiency, cleaner technologies, and sustainable production practices should complement early industrialization efforts. Policies should also focus on minimizing the risk of emissions-intensive industrial activities outsourced to less regulated economies, exacerbating global carbon leakage.

Furthermore, the observed U-shaped emissions trajectories in multiple sectors underscore the risks of premature deindustrialization and the environmental costs of globalization. Strengthening international cooperation on green technology transfers, sustainable trade policies, and environmental accountability mechanisms will ensure that economic complexity contributes to emissions reductions rather than exacerbating environmental challenges.

Overall, these findings demonstrate that sectoral complexity plays a pivotal role in shaping emissions trajectories, but its effects are highly heterogeneous. A more granular, sector-specific approach to sustainability policymaking is essential to mitigate environmental degradation while fostering technologically sophisticated and environmentally responsible economic growth.

\begin{table}[H]
	\centering
	\caption{Lower middle income and Low income Regression Coefficients Results.}
\begin{adjustbox}{width=\textwidth,center}
      \begin{tabular}{lllllllllll}
      \toprule
       &  & \multicolumn{9}{c}{Lower\ middle\ income} \\
       & Quantile & 0.1 & 0.2 & 0.3 & 0.4 & 0.5 & 0.6 & 0.7 & 0.8 & 0.9 \\
      Dependent Var & Independent Vars &  &  &  &  &  &  &  &  &  \\
      \midrule
      \multirow[t]{31}{*}{CO\textsubscript{2}\ emissions\ (metric\ tons\ per\ capita)\textsubscript{log}} & Agriculture\ SCI & 1.422*** & 1.252*** & 0.757* & 0.131 & 0.039 & 0.038 & 0.142 & -0.103 & -1.068*** \\
      & Agriculture\ SCI\textsuperscript{2} & -2.2*** & -1.774*** & -1.188* & -0.174 & 0.005 & -0.121 & -0.381 & -0.091 & 1.613*** \\
      & Electronics\ \&\ Instruments\ SCI & 0.563 & 0.676 & 0.296 & -0.15 & -0.351 & -0.386 & -0.636* & -0.454 & -0.602 \\
      & Electronics\ \&\ Instruments\ SCI\textsuperscript{2} & -0.988 & -0.991 & -0.754 & -0.267 & -0.262 & -0.163 & 0.025 & -0.192 & 0.38 \\
      & Fishing\ SCI & -0.687 & -1.288*** & -0.795* & -0.682* & -0.351 & -0.291 & -0.029 & -0.383 & -0.704** \\
      & Fishing\ SCI\textsuperscript{2} & 1.696 & 2.546** & 1.313 & 0.807 & 0.173 & -0.156 & -0.689 & -0.228 & 0.646 \\
      & Food\ \&\ Beverages\ SCI & 0.286 & 0.026 & 0.114 & 0.226 & 0.013 & 0.264 & 0.297 & 0.557 & 0.17 \\
      & Food\ \&\ Beverages\ SCI\textsuperscript{2} & -1.559 & -1.272 & -1.256 & -1.202 & -0.325 & -0.666 & -0.698 & -0.825 & -0.347 \\
      & Iron\ \&\ Steel\ SCI & -1.889*** & -1.275*** & -0.761* & -0.42 & -0.316 & -0.116 & 0.169 & 0.422 & 0.993** \\
      & Iron\ \&\ Steel\ SCI\textsuperscript{2} & 1.912* & 1.051* & 0.223 & -0.249 & -0.892* & -1.032* & -1.293*** & -1.102* & -1.243* \\
      & Machinery\ SCI & -0.767 & -0.845 & -0.564 & 0.603 & 0.762 & 0.462 & 0.173 & 0.045 & -0.38 \\
      & Machinery\ SCI\textsuperscript{2} & 2.891 & 2.994 & 2.602 & -1.226 & -0.859 & 0.73 & 0.855 & 1.468 & 1.606 \\
      & Metal\ Products\ SCI & -1.082 & -0.476 & -1.053 & -1.334* & -1.21** & -0.591 & -0.988* & -1.009* & -0.199 \\
      & Metal\ Products\ SCI\textsuperscript{2} & 0.548 & -0.15 & -0.306 & 0.282 & 1.843 & 0.842 & 3.268** & 4.482*** & 1.684 \\
      & Mining\ \&\ Quarrying\ SCI & -0.781** & -1.125*** & -0.509 & -0.296 & -0.117 & 0.005 & 0.04 & 0.143 & 0.06 \\
      & Mining\ \&\ Quarrying\ SCI\textsuperscript{2} & 1.248** & 1.467*** & 0.657 & 0.293 & 0.081 & 0.065 & 0.115 & 0.06 & 0.207 \\
      & Other\ Manufacturing\ SCI & 1.036 & 0.746 & 1.358** & 1.716*** & 1.74*** & 1.516*** & 1.552*** & 1.851*** & 1.184** \\
      & Other\ Manufacturing\ SCI\textsuperscript{2} & -0.323 & -0.049 & -0.43 & -0.618 & -0.588 & -0.454 & -0.676 & -1.225* & -0.55 \\
      & Petroleum,\ Chemicals\ \&\ Non-Metals\ SCI & 1.819*** & 1.752*** & 1.274* & 0.779 & 0.561 & 0.026 & -0.035 & -0.772 & -1.087* \\
      & Petroleum,\ Chemicals\ \&\ Non-Metals\ SCI\textsuperscript{2} & -1.633 & -1.737 & -0.536 & 0.469 & 0.216 & 0.508 & 0.27 & 0.681 & 1.012 \\
      & Textiles\ \&\ Wearing\ Apparel\ SCI & 0.486 & 1.131*** & 0.663 & 0.866** & 0.522* & 0.516* & 0.313 & 0.142 & 0.527 \\
      & Textiles\ \&\ Wearing\ Apparel\ SCI\textsuperscript{2} & 0.154 & -0.696 & -0.266 & -1.011 & -0.463 & -0.424 & -0.197 & 0.158 & -0.592 \\
      & Wood\ \&\ Paper\ SCI & 1.385** & 1.023* & 0.782 & 0.048 & -0.059 & -0.231 & -0.136 & -0.5 & 0.072 \\
      & Wood\ \&\ Paper\ SCI\textsuperscript{2} & -2.143* & -1.543 & -1.332 & 0.318 & 0.555 & 1.033 & 0.914 & 1.846** & 0.689 \\
      & Economic\ Globalisation & 0.693*** & 0.526*** & 0.564*** & 0.577*** & 0.558*** & 0.528*** & 0.563*** & 0.633*** & 0.627*** \\
      & Energy\ Consumption\ per\ Capita & 6.031*** & 5.952*** & 6.245*** & 6.175*** & 6.679*** & 6.654*** & 6.942*** & 6.519*** & 6.411*** \\
      & Environmental\ Patents\ per\ Capita & 0.518 & 0.3 & -1.943 & -1.874 & -2.648 & -0.9 & -1.313 & -2.912 & -3.856 \\
      & Intercept & -0.355*** & -0.23** & -0.104 & 0.042 & 0.051 & 0.06 & 0.075 & 0.178** & 0.419*** \\
      & Renewable\ energy\ consumption\ (\%\ of\ total\ final\ energy\ consumption) & -1.967*** & -1.915*** & -1.943*** & -1.908*** & -1.824*** & -1.755*** & -1.76*** & -1.795*** & -1.833*** \\
      & Total\ natural\ resources\ rents\ (\%\ of\ GDP) & 0.897*** & 0.71*** & 0.775*** & 0.55*** & 0.418*** & 0.616*** & 0.69*** & 0.759*** & 0.748*** \\
      & Urban\ population\ (\%\ of\ total\ population) & -0.064 & 0.118 & 0.056 & 0.028 & 0.048 & 0.025 & -0.015 & -0.014 & -0.119 \\
       \toprule
            &  & \multicolumn{9}{c}{Low\ income} \\
            & Quantile & 0.1 & 0.2 & 0.3 & 0.4 & 0.5 & 0.6 & 0.7 & 0.8 & 0.9 \\
      Dependent Var & Independent Vars &  &  &  &  &  &  &  &  &  \\
      \midrule
      \multirow[t]{31}{*}{CO\textsubscript{2}\ emissions\ (metric\ tons\ per\ capita)\textsubscript{log}} & Agriculture\ SCI & -0.398 & -0.797 & 0.355 & 0.335 & 0.228 & 1.256* & 0.326 & -0.404 & 0.254 \\
      & Agriculture\ SCI\textsuperscript{2} & 3.244* & 3.643** & 1.354 & 1.99 & 2.171 & -0.159 & 0.715 & 1.15 & -0.018 \\
      & Electronics\ \&\ Instruments\ SCI & 1.405 & -2.087 & -3.179** & -3.017** & -4.094*** & -1.93 & -3.14* & -4.628** & -4.364* \\
      & Electronics\ \&\ Instruments\ SCI\textsuperscript{2} & -29.418*** & -3.465 & 7.792* & 6.51 & 8.143 & 1.635 & 3.645 & 4.21 & 1.326 \\
      & Fishing\ SCI & -2.219*** & -2.724*** & -2.502*** & -2.72*** & -2.924*** & -2.74*** & -2.795*** & -2.837*** & -2.948*** \\
      & Fishing\ SCI\textsuperscript{2} & 3.975* & 5.381*** & 4.692*** & 5.422*** & 6.057*** & 5.842*** & 5.54*** & 6.047*** & 7.427*** \\
      & Food\ \&\ Beverages\ SCI & -1.449* & -0.934 & -1.768** & -1.377* & -1.521* & -2.353*** & -2.417*** & -2.511*** & -3.433*** \\
      & Food\ \&\ Beverages\ SCI\textsuperscript{2} & 3.107 & 0.915 & 3.589* & 2.312 & 3.887* & 5.987** & 5.455** & 5.527** & 7.001*** \\
      & Iron\ \&\ Steel\ SCI & 2.485** & 2.386*** & 1.984** & 1.206 & 0.343 & 0.78 & 0.785 & 0.234 & 0.276 \\
      & Iron\ \&\ Steel\ SCI\textsuperscript{2} & -4.686*** & -3.004* & -2.885* & -1.532 & 0.457 & -2.767 & -1.594 & -0.642 & 0.37 \\
      & Machinery\ SCI & -5.258** & -2.103 & -3.132* & -3.158 & -2.738 & -5.795** & -8.215*** & -7.733*** & -5.974** \\
      & Machinery\ SCI\textsuperscript{2} & -6.6 & -36.502** & -22.976* & -21.206 & -21.602 & -9.904 & 29.64* & 26.242 & 12.825 \\
      & Metal\ Products\ SCI & 0.901 & 0.982 & 1.898 & 1.893 & 3.312** & 0.528 & 2.256 & 3.096* & 4.041** \\
      & Metal\ Products\ SCI\textsuperscript{2} & -9.038 & -6.787 & -16.899* & -18.075** & -23.599*** & -4.991 & -9.462 & -8.981 & -14.024 \\
      & Mining\ \&\ Quarrying\ SCI & -2.108*** & -1.477*** & -2.316*** & -1.754*** & -1.888*** & -1.392*** & 0.184 & 1.383*** & 0.382 \\
      & Mining\ \&\ Quarrying\ SCI\textsuperscript{2} & 1.588* & 1.488* & 4.153*** & 3.311*** & 3.956*** & 4.282*** & 1.865** & 0.206 & 0.202 \\
      & Other\ Manufacturing\ SCI & -1.685 & 1.142 & 3.235*** & 3.502*** & 4.071*** & 5.368*** & 5.424*** & 5.534*** & 6.333*** \\
      & Other\ Manufacturing\ SCI\textsuperscript{2} & 5.979* & 0.674 & -2.606* & -2.612* & -3.152* & -5.342*** & -4.808** & -4.906*** & -4.907** \\
      & Petroleum,\ Chemicals\ \&\ Non-Metals\ SCI & 2.937** & 1.019 & 1.421 & 0.858 & 0.298 & 0.09 & -1.119 & -0.424 & -1.38 \\
      & Petroleum,\ Chemicals\ \&\ Non-Metals\ SCI\textsuperscript{2} & 0.44 & 2.99 & -0.838 & 2.244 & 2.495 & 0.993 & 1.774 & 0.579 & 6.771*** \\
      & Textiles\ \&\ Wearing\ Apparel\ SCI & 0.253 & 0.7 & 0.69 & 1.047* & 1.116* & 0.973* & 0.812 & 1.031* & 1.62*** \\
      & Textiles\ \&\ Wearing\ Apparel\ SCI\textsuperscript{2} & 3.35** & 0.841 & 0.176 & -0.517 & -0.432 & -1.035 & -0.892 & -1.23 & -3.26** \\
      & Wood\ \&\ Paper\ SCI & 2.056*** & 2.444*** & 2.971*** & 3.132*** & 3.183*** & 2.645*** & 2.514*** & 1.306* & 1.465 \\
      & Wood\ \&\ Paper\ SCI\textsuperscript{2} & -0.695 & -2.049* & -3.728*** & -4.521*** & -4.65*** & -4.733*** & -4.48*** & -3.116** & -4.001 \\
      & Economic\ Globalisation & 1.644*** & 1.53*** & 1.676*** & 1.704*** & 1.774*** & 1.784*** & 1.732*** & 1.591*** & 1.579*** \\
      & Energy\ Consumption\ per\ Capita & 6.459*** & 6.438*** & 5.71*** & 4.929*** & 4.57*** & 6.008*** & 6.922*** & 12.079*** & 18.84*** \\
      & Environmental\ Patents\ per\ Capita & -0.116 & -5.074 & -8.777** & -9.626** & -9.288* & -9.487 & -8.546 & -12.492 & -20.602 \\
      & Intercept & -1.56*** & -1.181*** & -1.201*** & -1.251*** & -1.35*** & -1.545*** & -1.498*** & -1.575*** & -1.735*** \\
      & Renewable\ energy\ consumption\ (\%\ of\ total\ final\ energy\ consumption) & -2.313*** & -2.645*** & -2.675*** & -2.692*** & -2.591*** & -2.366*** & -2.288*** & -2.143*** & -2.064*** \\
      & Total\ natural\ resources\ rents\ (\%\ of\ GDP) & -0.766*** & -0.45*** & -0.246 & -0.008 & 0.65*** & 0.519** & 0.73*** & 0.938*** & 1.513*** \\
      & Urban\ population\ (\%\ of\ total\ population) & 2.448*** & 2.214*** & 2.249*** & 2.446*** & 2.589*** & 2.861*** & 2.87*** & 2.725*** & 2.531*** \\
      \cline{1-11}
      \bottomrule
      \label{tab:lm-li-coefficients}
      \end{tabular}
      \end{adjustbox}
      {\centering\tiny Note: * p\textless0.05, ** p\textless0.01, *** p\textless0.001\par}
      \hfill
\end{table}


\section{Conclusion}
This study has examined the complex relationship between sectoral economic complexity, economic development, and environmental sustainability by introducing the Sectoral Complexity Index (SCI). By refining the Environmental Kuznets Curve (EKC) hypothesis at the sectoral level, this analysis provides a more granular understanding of how different industries contribute to CO\textsubscript{2} emissions across various stages of economic development. Through a quantile regression analysis of data spanning 127 countries from 1995 to 2020, the findings highlight distinct sector-specific emissions trajectories and reveal the intricate interplay between economic sophistication and environmental outcomes.

The sectoral classification confirms clear distinctions among economic sectors. Low-complexity sectors, including \TE, \AG, and \FI, are primarily engaged in basic production processes, while intermediate-complexity sectors such as \FO, \MI, \OT, and \WO\ exhibit greater product sophistication and industrial integration. High-complexity sectors, including \IR, \ME, \EL, \MA, and \PE, demonstrate a pronounced reliance on advanced technological capabilities, with \MA\ and \PE\ representing the highest levels of sophistication. This transition from resource-based industries to knowledge-intensive production structures illustrates the role of economic development in shaping industrial sophistication. It validates the SCI's ability to capture sectoral economic complexity.

The regression analysis confirms that sectoral emissions dynamics vary significantly. The \EL\ sector exhibits a U-shaped relationship with CO\textsubscript{2} emissions, contradicting the EKC hypothesis due to high specialization in complex electronic components and the offshoring of low-complexity production. Conversely, the \IR\ and \ME\ sectors strongly support the EKC hypothesis with consistent behavior across quantiles and income groupings. The \MI\ sector follows a similar pattern, with results indicating that emissions reductions at higher complexity levels are influenced by energy intensity, economies of scale, and technological advancements. These findings reinforce the notion that sectoral sophistication does not inherently lead to lower emissions but is contingent on industry characteristics, energy sources, and efficiency improvements.

The analysis of income-level groupings further detailed these findings. High-income countries exhibited a decrease in emissions within the \IR\ sector as sophistication increased, aligning with the EKC hypothesis. The \MI\ sector also supported the EKC hypothesis, suggesting that the transition to lower emissions occurs at advanced stages of economic development. However, the \PE\ sector, mainly due to oil-exporting countries, did not conform to the EKC hypothesis, indicating a need for targeted research and policy interventions in resource-dependent economies. The results for upper-middle-income countries showed that the \IR\ and \MA\ sectors strongly support the EKC hypothesis, indicating a strong transition in environmental outcomes driven by these sectors at this developmental stage. Conversely, lower-middle-income countries showed increased emissions associated with initial industrial development in the \OT\ sector, while low-income countries exhibited relevant behavior in low-complexity sectors.

The study's findings offer several policy implications for promoting sustainable development. High-complexity sectors such as \PE, \MA, and \EL\ significantly influence CO\textsubscript{2} emissions due to their energy-intensive production processes. Policymakers should prioritize investment in cleaner production, efficiency improvements, and regulatory frameworks that mitigate the environmental impact of these industries. Incentivizing green technology adoption and supporting innovation in emissions-intensive sectors is essential to align industrial sophistication with sustainability goals. In lower- and middle-income economies, reducing reliance on low-complexity sectors such as \AG, \TE, and \MI\ through economic diversification can enhance productive capabilities while minimizing environmental degradation.

Encouraging sustainable practices in high-energy-intensity sectors such as \IR\ and \ME\ is also crucial, as these sectors validated the EKC hypothesis. Policymakers should promote energy efficiency and the adoption of cleaner production technologies in these sectors while supporting research into sustainable practices. Additionally, the unique challenges faced by resource-dependent sectors, particularly in oil-exporting countries, require tailored interventions to reduce emissions and diversify into higher-complexity products without hindering economic growth.

Promoting sustainable practices in energy-intensive industries such as \IR\ and \ME\ is equally crucial. Given that these sectors conform to the EKC hypothesis, policies should focus on accelerating the transition to energy-efficient technologies and fostering research and innovation in sustainable manufacturing. Additionally, resource-dependent economies, particularly oil-exporting nations, require tailored interventions to reduce emissions while diversifying into higher-complexity industries without disrupting economic stability.

While local manufacturing has the mid-term positive effect of promoting employment and technology transfer in developing countries, it also has adverse effects by hindering economic independence and causing local environmental pollution. The \EL\ sector is well suited for local production because it is cheaper to set up a production base, and the products have a shorter life cycle than machinery and other sophisticated manufacturing sectors. Furthermore, the increasingly relevant role that the electronics and digital sectors play in the global economy makes this alternative even more compelling. Still, from a global environmental point of view, multinational manufacturing means CO\textsubscript{2} leakage from developed countries. It is an issue that should be discussed in a multilateral framework between developed and developing countries.

In conclusion, this study underscores the need for sector-specific policies to achieve sustainable economic development. By recognizing the diverse impacts of sectoral complexity on CO\textsubscript{2} emissions, policymakers can implement targeted interventions that support a transition toward higher-complexity, lower-emission industries. As countries continue to strive toward environmental sustainability, the insights provided by this study offer valuable guidance for balancing economic growth and the consequent environmental impacts, ensuring that economic complexity contributes to global sustainability efforts.

\subsection{Limitations and Future Research}
This study introduces the Sectoral Complexity Index (SCI), enabling a more nuanced analysis of the relationship between economic sectors and environmental indicators. By disaggregating economic complexity to the sectoral level, the SCI provides deeper insights into the heterogeneous pathways through which different sectors influence environmental outcomes, offering a robust framework for understanding the economic-environment nexus. The study adopts a robust yet general approach, incorporating quantile regression analysis and relevant control variables to ensure the reliability of the findings. The results highlight significant patterns in the relationship between sectoral sophistication and CO\textsubscript{2} emissions, from which further research can be derived. For example, scholars can expand on analyzing the relationship with other environmental or economic indicators, propose different models to account for other interactions, or dig deeper into the research of individual sectors by introducing economic complexity tools such as relatedness and the product space.

The study also has certain limitations that provide opportunities for future research. A key limitation lies in the Economic Complexity framework, which does not account for digitization, Artificial Intelligence (AI), Large Language Models (LLMs), or software-based services and products, which are increasingly influential in shaping modern economies and their environmental impacts. Customs do not govern international trade of these services and products, so thorough data is not readily available. \cite{stojkoskiMultidimensionalEconomicComplexity2023} addressed this limitation and introduced the Multidimensional ECI, which showed promising results. Future studies could address this gap by incorporating the Multidimensional ECI or by including alternative metrics or datasets that capture the complexity and environmental implications of digital and service-based economic activities. Expanding the SCI framework to include these dimensions would enhance its applicability and provide a more comprehensive understanding of the interplay between economic sophistication and environmental sustainability.


\bibliographystyle{abbrvnat}
\bibliography{library}      

\newpage

\section{Author Contributions}
All authors jointly supervised and contributed to this work.

\section{Data Availability}
The raw exports datataset is not available due to licensing agreements with the Observatory of Economic Complexity\footnote{\url{https://oec.world}}. 
The datasets generated during and/or analyzed during the study are available in a correponding Figshare public repository\footnote{\url{https://figshare.com/s/6ce82e9a10af21ddc774}}. A public code repository\footnote{\url{https://github.com/montanon/sectoral_complexity}} is available on publication with functionality to replicate this study. Further inquires can be made with the corresponding author.

\section{Competing Interests}
The authors declare no competing interests.

\section{Ethical Approval}
This article does not contain any studies with human participants performed by any of the authors.

\section{Informed Consent}
This article does not contain any studies with human participants performed by any of the authors.

\section{Figures Legends}

\begin{itemize}
      \item \textbf{Figure 1:} \textbf{Economic sectors products PCI distribution.} Histogram of the PCI values of all products assigned to each economic sector with samples of corresponding products at distinct 5\%, 50\%, and 95\% quantiles. The distribution of PCI values varies by sector, with Panels b(\EL), f(\MA), and j(\PE) having high PCI, and Panels a(\AG), c(\FI), and k(\TE) having low PCI. Panel h(\MI) has lower than average PCI, while Panels e(\IR) and g(\ME), whose products are dervide from h, have higher than average PCI. Likewise, Panel d(\FO) has lower than average PCI but higher than Panels a and c. Panel l(\WO) has slightly above average PCI, while Panel i(\OT) has average PCI.
      \item \textbf{Figure 2:} \textbf{Economic sectors SCI vs CO\textsubscript{2} emissions.} Scatter plots of SCI values with respect to CO\textsubscript{2} emissions for all 127 countries between 1995 and 2020, disaggregated by sector. Data points are coloured based on the income level of the respective country at the given year. The relationship between SCI and emissions varies by sector. Panels a(\AG), c(\FI), and k(\TE) show a weak correlation between emissions and SCI. Panels b(\EL), e(\IR), f(\MA), g(\ME), and j(\PE) show a strong correlation between income and SCI. Panels d(\FO), h(\MI), and l(\WO) also show a strong correlation, though weaker than in Panels b, e, f, g, and j. Panel i(\OT) shows weak correlation with a wide spread of values. The EKC hypothesis cannot be identified from this figure as CO\textsubscript{2} emissions are total emissions by country and do not represent the relationship derived from partial regression coefficients.
      \item \textbf{Figure 3:} \textbf{Countries SCI vs CO\textsubscript{2} emissions by econmic sector for 2020.} Scatter plots of SCI with respect to CO\textsubscript{2} emissions for all 127 countries by sector in 2020. Each data point is represented as the respective country flag to illustrate distinctive behavior by country as captured by the SCI. Panels a(\AG), b(\EL), c(\FI), d(\FO), e(\IR), f(\MA), g(\ME), h(\MI), i(\OT), j(\PE), k(\TE), and l(\WO) show diverse patterns by sector, with country-specific characteristics expressed as varying degrees of sophistication across the portayed economic sectors.
\end{itemize}

% \newpage
% \include{supplementary_material}

\end{document}